\section[Diskrete Verteilungen]{Spezielle diskrete Verteilungen}

\begin{enumerate}
\item Ein W\"{u}rfel wird f\"{u}nf Mal geworfen. Sei $X$ die
Anzahl der Sechsen.
\begin{enumerate}
\item Wie ist $X$ verteilt?
\item Wie gro\ss\ ist die Wahrscheinlichkeit, dass genau zwei Sechsen
geworfen werden?
\item Wie gro\ss\ ist die Wahrscheinlichkeit, dass mehr als eine Sechs
geworfen wird?
\end{enumerate}

\item In einer Computer-Chip-Fabrik soll die Qualit\"{a}%
tskontrolle verbessert werden. An einer bestimmten Maschine ist der
Ausschussanteil selbst bei optimaler Einstellung 20\% (bei nicht optimaler
Einstellung ist der Anteil noch h\"{o}her). Aus der laufenden Produktion
werden nun in unregelm\"{a}ssigen Abst\"{a}nden 10 Chips entnommen und gepr%
\"{u}ft. Bei der letzten \"{U}berpr\"{u}fung ergab sich, dass von den 10
Chips 5 kaputt waren. W\"{u}rden Sie die Produktion stoppen lassen, um die
Maschine neu zu justieren?

\item Das folgende Zitat ist aus R.A. Fisher (1960), \emph{The
Design of Experiments}, 7. Aufl., S. 11: \textquotedblleft A Lady declares
that by tasting a cup of tea made with milk she can discriminate whether the
milk or the tea infusion was first added to the cup.\textquotedblright\ Um
die Behauptung zu \"{u}berpr\"{u}fen, reichen Sie der Lady 10 Tassen, in die
teils zuerst die Milch, teils erst der Tee eingegossen wurde. Die Lady gibt
in 8 F\"{a}llen die richtige Antwort. Wie gro\ss\ ist die Wahrscheinlichkeit
dieses Ergebnisses, wenn die Lady in Wirklichkeit nur r\"{a}t (also mit
jeweils 50\% Wahrscheinlichkeit sagt, dass erst der Tee bzw. erst die Milch
eingegossen wurde)? Wie gro\ss\ ist die Wahrscheinlichkeit eines
schlechteren Ergebnisses?

\item Unter der Hotline-Nummer einer Registrierkassenfirma
rufen im Durchschnitt tags\"{u}ber 20 Leute pro Stunde an. Wie gro\ss\ ist
die Wahrscheinlichkeit, dass in einer Viertelstunde genau 2 Personen anrufen?

In der Zeit zwischen 10 Uhr abends und 6 Uhr morgens rufen durchschnittlich
nur 4 Personen an. Wie gro\ss\ ist die Wahrscheinlichkeit, dass w\"{a}hrend
der Nachtschicht niemand anruft?

\item Die Wahrscheinlichkeit, dass im \"{o}ffentlichen
Nahverkehr einer Stadt eine Fahrkartenkontrolle durchgef\"{u}hrt wird, ist
bei einer zuf\"{a}llig ausgew\"{a}hlten Fahrt 2\%. Nehmen Sie an, dass die
Kontrollereignisse verschiedener Fahrten stochastisch unabh\"{a}ngig
voneinander sind.
\begin{enumerate}
\item Wie gro\ss\ ist die Wahrscheinlichkeit, dass man genau bei der zehnten
Fahrt kontrolliert wird?
\item Wie gro\ss\ ist die Wahrscheinlichkeit, dass man mindestens 30 Fahrten
lang nicht kontrolliert wird?
\end{enumerate}

\item An einem Drive-In-Restaurant halten im Schnitt 15 Autos pro
Stunde. Gehen Sie davon aus, dass die Anzahl der Autos pro Stunde
Poisson-verteilt ist. Die Wahrscheinlichkeit, dass in einem Auto nur eine
Person sitzt, ist 0.5. Die Wahrscheinlichkeit, dass in einem Auto zwei
Personen sitzen, ist 0.3. Die Wahrscheinlichkeit, dass in einem Auto drei
Personen sitzen, ist 0.1. Die Wahrscheinlichkeit, dass in einem Auto vier
Personen sitzen, ist auch 0.1. (Mehr als vier Personen pro Auto werden in
dem Restaurant nicht bedient.) Jede Person im Auto ist ein Kunde.
\begin{enumerate}
\item Wie viele Kunden kommen durchschnittlich innerhalb einer halben
Stunde? Hinweis: Wenn $X$ und $Y$ unabh\"{a}ngig sind, gilt $E(XY)=E(X)E(Y)$.
\item Wie gro\ss\ ist die Wahrscheinlichkeit, dass innerhalb von 20 Minuten
mehr als 4 Autos kommen?
\item Wie gro\ss\ ist die Wahrscheinlichkeit, dass innerhalb von 4 Minuten
genau 3 Personen bedient werden?
\end{enumerate}
\end{enumerate}
