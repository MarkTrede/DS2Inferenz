\newcommand{\woche}{}
\newcommand{\nr}{1}
\newcommand{\thema}{}

\documentclass[12pt]{scrartcl}
\usepackage{amssymb}
\usepackage[utf8]{inputenc}
\usepackage{amsmath}
\usepackage[ngerman]{babel}
\usepackage{graphicx}
\usepackage{diagbox}
\usepackage{bbm}
\usepackage[T1]{fontenc}
\usepackage{comment}
\usepackage{tikz}
\usepackage{slashed}
\setlength{\parindent}{0pt}

\usepackage[lastexercise]{exercise}%noanswer

\renewcommand{\ExerciseHeader}{{\medskip\noindent\textbf{\normalsize
                \ExerciseName\ \ExerciseHeaderNB\ExerciseHeaderTitle
               \ExerciseHeaderOrigin\medskip\\}}}

\renewcommand{\AnswerHeader}{{\bigskip\noindent \textbf{\normalsize Lösung zu \ExerciseName\ \ExerciseHeaderNB\medskip\\}}}

\makeatletter
\renewcommand{\@@@ExeEnv}{%
    \vskip\ExerciseSkipBefore \@QuestionLevel1 \refstepExecounter%
    \begingroup%
        \@getExerciseInfo\ExerciseHeader%
    \endgroup%
}
\makeatother

\renewcommand{\QuestionNB}{\alph{Question})\ }
\renewcommand{\subQuestionNB}{\roman{subQuestion})\ }
\renewcommand{\subsubQuestionNB}{\arabic{subsubQuestion})\ }

\addto\captionsngerman{%
 \def\listexercisename{Liste der {\"U}bungen}%
 \def\ExerciseName{Aufgabe}%
 \def\AnswerName{L{\"o}sung zu Aufgabe}%
 \def\ExerciseListName{{\"U}b.}%
 \def\AnswerListName{L{\"o}sung}%
 \def\ExePartName{Teil}%
 \def\ArticleOf{von\ }%
 }
\usepackage[pdftex]{hyperref}
\hypersetup{
    pdftitle = {Hausaufgabe \nr},
    pdfauthor = {me}
}




\begin{document}
\noindent
{\small{\textbf {Data Science 2}} (Wintersemester 23/24)\hfill \woche \\
Prof.~Dr.~Mark Trede\\
Fabian Apostel B.~Sc.\\
Daniel Stroth B.~Sc.}
\vspace{0.1cm}
\hrule


\vspace{0,5cm}
\begin{center}
{\Large{\textbf {Hausaufgabe \nr}}}\\[0.3cm] 
{\small{\textit {\thema}}}\\[1cm]
\end{center}

\begin{Exercise}
Sei $f\colon\left[0,\infty\right)\to\mathbb{R}$ eine stetige Funktion. Sei $F\colon \left[0,\infty\right)\to \mathbb{R}$ definiert durch $F(x)=\int_0^xf(t)\operatorname{d}t$ für $x>0$.
\Question Zeigen Sie, dass für alle $a>0$ gilt:
\begin{equation*}
	\int_0^{a}xf(x)\operatorname{d}x=aF(a)-\int_0^{a}F(x)\operatorname{d}x
\end{equation*}
\Question Sei $\lambda>0$. Die Funktion $f$ sei nun gegeben durch $f(x)=\lambda e^{-\lambda x}$ für $x\geq 0$. Berechnen Sie 
\begin{equation*}
	\int_0^\infty xf(x)\operatorname{d}x.
\end{equation*}
Gehen Sie hierbei davon aus, dass $\lim_{a\to\infty}ae^{-\lambda a}=0$
\Question Seien nun $k,\lambda >0$ und $f(x):=\lambda\cdot k\cdot  (\lambda x)^{k-1}e^{-(\lambda x)^k}$ für $x\geq 0$. Berechnen Sie
\begin{equation*}
	\int_0^xf(t)\operatorname{d}t.
\end{equation*}
\end{Exercise}
\begin{Answer}
\Question Man muss die Voraussetzungen des Hauptsatzes der Differential- und Integralrechnung (HDI) nachprüfen. Dafür überprüft man, ob die Ableitung beider Seiten der gesuchten Gleichung gleich sind. Für die linke Seite gilt:
\begin{equation*}
	\left(\int_0^{a}xf(x)\operatorname{d}\right)^{'}\stackrel{\text{HDI}}{=} af(a) 
\end{equation*}
Für die rechte Seite gilt:
\begin{eqnarray*}
	\left(aF(a)-\int_0^{a}F(x)\operatorname{d}x\right)^{'}&\stackrel{\text{Summenregel}}{=}& \left(aF(a)\right)^{'}-\left(\int_{0}^{a}F(x)\operatorname{d}x\right)^{'} \\
	&\stackrel{\text{Produktregel und HDI}}{=}&F(a)+aF^{'}(a)-F(a)\\
	&\stackrel{\text{HDI}}{=}&af(a).
\end{eqnarray*}
\Question Nach der Aufleitungsregel für die Exponentialfunktion wissen wir, dass 
$F(x)=\int_0^{x}\lambda e^{-\lambda t}\operatorname{d}t=\left[-e^{-\lambda t}\right]_0^x=1-e^{-\lambda x}$ ist.
Mit a) gilt nun:
\begin{eqnarray*}
	\int_0^{\infty}xf(x)\operatorname{d}x&=&\lim_{a\to\infty}\int_0^{a}xf(x)\operatorname{d}x\\
	&\stackrel{\text{a)}}{=}&\lim_{a\to\infty}\left(aF(a)-\int_0^{a}F(x)\operatorname{d}x\right)\\
	&=&\lim_{a\to\infty}\left(a\cdot(1-e^{-\lambda a})-\int_0^{a}1-e^{-\lambda x}\operatorname{d}x\right)\\
	&=&\lim_{a\to\infty}\left(a\cdot(1-e^{-\lambda a})-\int_0^{a}1\operatorname{d}x+\int_0^{a}e^{-\lambda x}\operatorname{d}x\right)\\
	&=&\lim_{a\to\infty}\left(a\cdot(1-e^{-\lambda a})-a+\left[-\frac{1}{\lambda}e^{-\lambda x}\right]_0^a\right)\\
	&=&\lim_{a\to\infty}\left(-ae^{-\lambda a}+\left[-\frac{1}{\lambda}e^{-\lambda x}\right]_0^a\right)\\
	&=&\lim_{a\to\infty}\left(-ae^{-\lambda a}-\frac{1}{\lambda}e^{-\lambda a}+\frac{1}{\lambda}\right)\\
	&=&\lim_{a\to\infty}-ae^{-\lambda a}-\lim_{a\to\infty}\frac{1}{\lambda}e^{-\lambda a}+\lim_{a\to\infty}\frac{1}{\lambda}\\
	&=&\frac{1}{\lambda}
\end{eqnarray*}
\Question Dieses Integral muss durch Substitution berechnet werden. Dafür setzen wir $u(t)=\lambda^kt^k$ damit ergibt sich $\dfrac{\operatorname{d}u(t)}{\operatorname{d}t}=\lambda^k \cdot k\cdot t^{k-1}\Leftrightarrow \operatorname{d}t=\frac{1}{\lambda^k\cdot k\cdot t^{k-1}}\operatorname{d}u$
Es ergibt sich dann folgende Rechnung:
\begin{eqnarray*}
	\int_0^{x}f(t)\operatorname{d}&=&\int_{0}^{x}\lambda\cdot k\cdot (\lambda t)^{k-1}e^{-(\lambda t)^k}\operatorname{d}t\\
	&\stackrel{\text{Substitution}}{=}&\int_0^{\lambda^kx^k} \lambda\cdot k\cdot (\lambda t)^{k-1}e^{-u}\frac{1}{\lambda^k\cdot k\cdot t^{k-1}}\operatorname{d}u\\
	&=&\int_0^{\lambda^kx^k}e^{-u}\operatorname{d}u\\
	&=&\left[-e^{-u}\right]_0^{\lambda^kx^k}\\
	&=&1-e^{-\lambda^kx^k}
\end{eqnarray*}
\end{Answer}


\begin{Exercise} Sie dürfen in dieser Aufgabe nur die üblichen Rechenregeln für reelle Zahlen verwenden, sowie die drei Axiome für Wahrscheinlichkeiten. Die Axiome lauten:
	\begin{itemize}
		\item[I]  Nichtnegativität: $P(A)\geq0$ für alle Ereignisse $A$,
		\item[II] Normiertheit: $P(\Omega)=1$,
		\item[III] Additivität: Für disjunkte Ereignisse $A$ und $B$ (d.h. $A\cap B=\emptyset$) gilt:
		\begin{equation*}
			P(A\cup B)=P(A)+P(B).
		\end{equation*}
	\end{itemize}
	Sie dürfen außerdem bereits bewiesene Rechenregeln in dieser Aufgabe für den Beweis der anderen Rechenregeln verwenden.
	Seien $A$, $B$ und $C$ drei Ereignisse des Ereignisraums $\Omega$ ($A,B,C\subset \Omega$). 
	Beweisen Sie die folgenden Rechenregeln:
	\Question $P(\overline{A})=1-P(A)$,
	\Question $P(\emptyset)=0$,
	\Question $0\leq P(A)\leq1$,
	\Question $P(A\setminus B)=P(A)-P(A\cap B)$. Wie verändert sich diese Regel, wenn $B\subset A$?
	\Question $P(A\cup B)=P(A)+P(B)-P(A\cap B)$,
	\Question $P(A \cup B \cup C )=P(A)+P(B)+P(C)-P(A\cap B)-P(A\cap C)-P(B\cap C)+P(A\cap B\cap C)$.
\end{Exercise}

\begin{Answer}
\Question 
\begin{eqnarray*}
	1\stackrel{\text{II}}{=}P(\Omega)=P(A\cup\overline{A})\stackrel{\text{III}}{=}P(A)+P(\overline{A}).
\end{eqnarray*}
Das ist äquivalent zu der Aussage $1-P(A)=P(\overline{A})$.
\Question 
\begin{eqnarray*}
	P(\emptyset)\stackrel{\text{a)}}{=}1-P(\overline{\emptyset})=1-P(\Omega)\stackrel{\text{II}}{=}1-1=0.
\end{eqnarray*}
\Question Das erste $\leq$ geht aus dem Axiom I hervor. Das zweite folgt aus der folgenden Rechnung:
\begin{eqnarray*}
1\stackrel{\text{II}}{=}P(\Omega)=P(A\cup\overline{A})\stackrel{\text{III}}{=}P(A)+P(\overline{A})\geq P(A).
\end{eqnarray*}
Dabei gilt das $\geq$, da nach dem Axiom I gilt, dass $P(\overline{A})\geq0$.
\Question 
\begin{eqnarray*}
	P(A)=P((A\setminus B)\cup (A\cap B))\stackrel{\text{III}}{=}P(A\setminus B)+P(A \cap B).
\end{eqnarray*}
Diese Aussage ist dann äquivalent zu der Aussage $P(A\setminus B)=P(A)-P(A\cap B)$.
Sobald $B\subset A$ gilt $A\cap B=B$ und die Rechenregel verändert sich zu $P(A\setminus B)=P(A)-P(B)$.
\Question 
\begin{eqnarray*}
	P(A\cup B)&=&P((A\setminus B)\cup(B\setminus A) \cup(A\cap B))\\
	&\stackrel{\text{III}}{=}&P((A\setminus B)+P(B\setminus A) +P(A\cap B))\\
	&\stackrel{\text{d)}}{=}&P(A)-P(A\cap B)+P(B)-P(A\cap B)+P(A\cap B)\\
	&=&P(A)+P(B)-P(A\cap B).
\end{eqnarray*}
\Question 
\begin{eqnarray*}
	P(A\cup B\cup C)&=&P\left((A\setminus (B\cup C))\cup (B\setminus (A\cup C))\cup (C\setminus(A\cup B))\right.\\
	&&\left.\cup ((A\cap B)\setminus C)\cup ((A\cap C)\setminus B)\cup ((B\cap C)\setminus A)\cup (A\cap B\cap C)\right)\\
	&\stackrel{\text{III}}{=}&P(A\setminus (B\cup C))+ P(B\setminus (A\cup C))+P(C\setminus(A\cup B))\\
	&&+ P((A\cap B)\setminus C)+ P((A\cap C)\setminus B)\\
	&&+ P((B\cap C)\setminus A)+ P(A\cap B\cap C)\\
	&\stackrel{c)}{=}&P(A)-P(A\cap(B\cup C))+P(B)-P(B\cap (A\cup C))\\
	&&+P(C)-P(C\cap (A\cup B))+P(A\cap B)-P(A\cap B \cap C)\\
	&& +P(A\cap C)-P(A\cap B \cap C)+P(B\cap C)-P(A\cap B \cap C)\\
	&&+ P(A\cap B\cap C)\\
	&=&P(A)+P(B)+P(C)-P(A\cap(B\cup C))-P(B\cap (A\cup C))\\
	&&-P(C\cap (A\cup B))+P(A\cap B)\\
	&& +P(A\cap C)+P(B\cap C)-2P(A\cap B\cap C)\\
\end{eqnarray*}
Um die Rechnung übersichtlicher zu gestalten erhalten betrachten wir folgende Nebenrechnung:
\begin{eqnarray*}
	P(A\cap (B\cup C))&=& P((A\cap B)\cup(A\cap C))\\
	&\stackrel{\text{e)}}{=}&P(A\cap B)+P(A\cap C)-P(A\cap B\cap C)
\end{eqnarray*}
Man kann nun die analoge Rechnung für $P(B\cap (A\cup C))$ und $P(C\cap (A\cup B))$ durchführen. Setzt man nun diese Werte in die Rechnung von oben ein ergibt sich:
\begin{eqnarray*}
	P(A\cup B\cup C)&=&P(A)+P(B)+P(C)\\
	&&-(P(A\cap B)+P(A\cap C)-P(A\cap B\cap C))\\
	&&-(P(B\cap A)+P(B\cap C)-P(A\cap B\cap C))\\
	&&-(P(B\cap C)+P(C\cap A)-P(A\cap B\cap C))\\
	&&+P(A\cap B)+P(A\cap C)+P(B\cap C)\\
	&&-2P(A\cap B\cap C)\\
	&=&P(A)+P(B)+P(C)\\
	&&-P(A\cap B)-P(A\cap C)-P(B\cap C)\\
	&&+P(A\cap B\cap C).
\end{eqnarray*} 
\end{Answer}

\begin{Exercise}
Im Fußball ist der Elfmeterschuss eine einfache spieltheoretische Situation für den Schützen ($S$) und den Torwart ($T$). Wir nehmen an, dass der Schütze entscheiden muss, ob er nach rechts ($S_R$) oder nach links ($S_L$) schießen wird, und dass es keine andere Möglichkeiten als rechts und links gibt. Gleichermaßen muss der Torwart entscheiden, ob er nach rechts oder links springt ($T_R$ oder $T_L$). Typischerweise tendieren die Schützen dazu auf eine bestimmte Seite des Tores zu schießen (meistens schießen Rechtsfüßer auf die rechte und Linksfüßer auf die linke Seite). Wir nennen die präferierte Seite des Schützen die typische und die andere die untypische Seite. Um die Notation einfach zu halten, und ohne Beschränkung der Allgemeinheit, nehmen wir an, dass die typische Seite die rechte Seite ist. 

Die Wahrscheinlichkeit, dass der Schütze ein Tor erzielt, hängt von der Entscheidung des Schützen und des Torwarts ab. Sollte der Schütze sich beispielsweise dazu entscheiden auf die rechte Seite zu schießen ($S_R$) und sollte der Torwart sich dazu entscheiden auf die linke Seite zu springen ($T_L$), so ist die Trefferwahrscheinlichkeit knapp 93\%. Die folgende Aufstellung gibt die bedingten Wahrscheinlichkeiten
\[
P(A|S_L\cap T_L),\quad P(A|S_R\cap T_L),\quad P(A|S_L\cap T_R),\quad P(A|S_R\cap T_R)
\]
an. Mit $A$ wird das Ereignis ``Treffer'' bezeichnet. Woher diese bedingten Wahrscheinlichkeiten kommen, ist in dieser Aufgabe unwichtig. Sie werden einfach als bekannt angenommen.

\begin{center}
	\begin{tabular}{cc|cc}
		&  & \multicolumn{2}{|c}{Torwart} \\
		&  & $T_L$ & $T_R$ \\ \cline{2-4}
		Schütze& $S_L$ & 0.5830 & 0.9497 \\
		& $S_R$ & 0.9291 & 0.6992
	\end{tabular}
\end{center}

Aus der Spieltheorie ist bekannt, dass es die beste Strategie für den Schützen ist, seine Schüsse zu randomisieren, das heißt, dass er mit einer bestimmten Wahrscheinlichkeit $s_L$ auf die linke Seite und mit $s_R=1-s_L$ auf die rechte Seite schießt.Auf die gleiche Weise ist die beste Strategie für den Torwart seine Entscheidung ebenfalls zu randomisieren (mit Wahrscheinlichkeiten $t_L$ und $t_R$). In der Spieltheorie nennt man diese Strategien gemischte Strategien.
Bestimmen Sie die optimale gemischte Strategie für den Schützen und den Torwart.

\textit{Hinweis: Im Gleichgewicht wählt der Torwart $t_L$ und $t_R=1-t_L$ so, dass -- gegeben die bedingten Wahrscheinlichkeiten in der obigen Tabelle -- die Trefferwahrscheinlichkeit für den Schützen gleich ist, unabhängig davon, ob er nach links oder rechts schießt. Gleichermaßen wählt der Schütze $s_L$ und $s_R=1-s_L$ so, dass der Torwart indifferent zwischen rechts und links ist.}
\end{Exercise}

\begin{Answer}
Wir bestimmen zuerst die gleichgewichte Wahrscheinlichkeit $t_L$ für den Torwart. Er wählt sie so, dass die Trefferwahrscheinlichkeit für den Schützen bei einem Schuss nach links genauso hoch ist wie bei einem Schuss nach rechts. Die Trefferwahrscheinlichkeit bei einem Schuss nach links beträgt
\begin{align*}
P(A|S_L)&=\frac{P(A\cap S_L)}{P(S_L)}\\
&=\frac{P((A\cap S_L\cap T_L)\cup (A\cap S_L\cap T_R))}{P(S_L)}\\
&=\frac{P(A\cap S_L\cap T_L)+P(A\cap S_L\cap T_R)}{P(S_L)}\\
&=\frac{P(A|S_L\cap T_L)P(S_L\cap T_L)+P(A|S_L\cap T_R)P(S_L\cap T_R)}{P(S_L)}
\end{align*}
Da die Entscheidungen von Schützen und Torwart unabhängig voneinander getroffen werden, gilt $P(S_L\cap T_L)=P(S_L)P(T_L)$ etc. Die Wahrscheinlichkeit $P(S_L)$ lässt sich heraus kürzen. Also ist
\begin{align*}
P(A|S_L)&=P(A|S_L\cap T_L)P(T_L)+P(A|S_L\cap T_R)P(T_R)\\
&= 0.5830 t_L+0.9497 (1-t_L)
\end{align*}
Auf gleiche Weise wird $P(A|S_R)$ bestimmt. Es ergibt sich
\[
P(A|S_R)=0.9291 t_L+0.6992 (1-t_L).
\]
Setzt man $P(A|S_L)$ und $P(A|S_R)$ gleich, ergibt sich die Gleichung
\[
0.5830 t_L+0.9497 (1-t_L)=0.9291 t_L+0.6992 (1-t_L).
\]
Auf\/lösen nach $t_L$ ergibt
\begin{align*}
t_L &= 0.4199\\
t_R &= 1-t_L= 0.5801
\end{align*}
Analog erhält man $s_L$ und $s_R$ durch Gleichsetzen von
$P(A|T_L)=P(A|T_R)$. Als Resultat ergibt sich
\begin{align*}
s_L &= 0.3854\\
s_R &= 1-s_L= 0.6146.
\end{align*}
\end{Answer}


\begin{Exercise}
	Gehen Sie von zwei Würfeln mit den Folgenden Augenzahlen aus:
	\begin{equation*}
		\text{Würfel 1}:6~~3~~3~~3~~3~~3;~~~~~~~\text{Würfel 2}:5~~5~~5~~2~~2~~2~~
	\end{equation*}
	Geben Sie eine Beschriftung für einen dritten Würfel so an, dass das folgende Spiel für den Spieler B vorteilhaft ist: Spieler A darf einen der drei Würfel wählen, dann darf Spieler B einen der verbleibenden Würfel wählen. Jeder würfelt mit dem von ihm gewählten Würfel. Derjenige, der die höhere Augenzahl hat, hat gewonnen. Gehen Sie hierbei davon aus, dass Spieler B in jedem Fall weiß, welcher Würfel vorteilhaft ist. Die Würfel sollen so beschriftet sein, dass ein \glqq unentschieden\grqq~ nicht möglich ist, d.h., dass der dritte Würfel nur mit den Augenzahlen 4 und 1 beschriftet werden kann. Gehen Sie außerdem davon aus, dass die Würfe unabhängig voneinander sind.\\
	\textit{Hinweis: Untersuchen Sie diesen Sachverhalt, indem Sie dieses Spiel in verschiedene Szenarien unterteilen.}\\
	
\end{Exercise}

\begin{Answer}
	Um die Notation zu vereinfachen führen wir folgende Notation ein:
	\begin{eqnarray*}
		A_i&:=&\left\lbrace \text{Spieler A wirft die Augenzahl i}\right\rbrace\\
		B_i&:=&\left\lbrace \text{Spieler B wirft die Augenzahl i}\right\rbrace\\
		G&:=& \left\lbrace\text{Spieler B gewinnt}\right\rbrace
	\end{eqnarray*}
	Wir teilen dieses Spiel in Szenarien auf: In jedem dieser Szenarien gilt, dass B einen Vorteil hat, wenn $P(G)>\frac{1}{2}$. Wir können darüber hinaus davon ausgehen, dass die beiden Würfelwürfe unabhängig voneinander stattfinden.\\
		\underline{Szenario 1: Spieler A wählt Würfel 2 und Spieler B wählt Würfel 1}\\		
		Wir benutzen zunächst den Satz der totalen Wahrscheinlichkeit:
		\begin{eqnarray*}
			P(G)&=&P(G\vert A_2)\cdot P(A_2)+P(G\vert A_5)\cdot P(A_5)\\
			&=&\frac{P(G\cap A_2)}{P(A_2)}\cdot P(A_2)+\frac{P(G\cap A_5)}{P(A_5)}\cdot P(A_5)\\
			&=&\frac{P( A_2)}{P(A_2)}\cdot P(A_2)+\frac{P(B_6\cap A_5)}{P(A_5)}\cdot P(A_5)\\
			&=&P(A_2)+\frac{P(B_6)\cdot P(A_5)}{P(A_5)}\cdot P(A_5)\\
			&=&P(A_2)+P(B_6)\cdot P(A_5)\\
			&=& \frac{1}{2}+\frac{1}{6}\cdot \frac{1}{2}\\
			&>& \frac{1}{2}.
		\end{eqnarray*}
		Unabhängig davon, wie Würfel 3 aussieht, könnte Spieler B also Würfel 1 wählen.\\
		\underline{Szenario 2: Spieler A wählt Würfel 1 und Spieler B wählt Würfel 3}\\
		Spieler B kann nur gewinnen, wenn Spieler A keine 6 gewürfelt hat. Da ein Unentschieden nicht geschehen darf, kann Würfel 3 nur die Augenzahlen 1 und 4 besitzen. Sei $b\in\left\lbrace 0,1,2,3,4,5,6\right\rbrace$ die Anzahl der Seiten von Würfel 3, die die Augenzahl 4 besitzen
			Wieder Satz der totalen Wahrscheinlichkeit:
			\begin{eqnarray*}
				P(G)&=&P(G\vert A_3)\cdot P(A_3)+P(G\vert A_6)\cdot P(A_6)\\
				&=&\frac{P(G\cap A_3)}{P(A_3)}\cdot P(A_3)+\frac{P(G\cap A_6)}{P(A_6)}\cdot P(A_6)\\
				&=&\frac{P(B_4\cap A_3)}{P(A_3)}\cdot P(A_3)+\frac{P(\emptyset)}{P(A_6)}\cdot P(A_6)\\
				&=&\frac{P(B_4)\cdot P(A_3)}{P(A_3)}\cdot P(A_3)+0\\
				&=& P(B_4)\cdot P(A_3)\\
				&=&\frac{b}{6}\cdot \frac{5}{6}\\
			\end{eqnarray*}
		Diese Wahrscheinlichkeit ist nur dann größer als $\frac{1}{2}$, wenn $b>3.6$ also $b\geq 4$ ist.\\
		Wir müssen sicherstellen, dass Würfel 3 vorteilhaft gegenüber Würfel 2 ist:\\
		\underline{Szenario 3: Spieler A wählt Würfel 3 und Spieler B wählt Würfel 2}\\
		Wir benutzen wiederholt den Satz der totalen Wahrscheinlichkeit:
		\begin{eqnarray*}
			P(G)&=&P(G\vert A_4)\cdot P(A_4)+P(G\vert A_1)\cdot P(A_1)\\
			&=&\frac{P(G\cap A_4)}{P(A_4)}\cdot P(A_4)+\frac{P(G\cap A_1)}{P(A_1)}\cdot P(A_1)\\
			&=&\frac{P(B_5\cap A_4)}{P(A_4)}\cdot P(A_4)+\frac{P(A_1)}{P(A_1)}\cdot P(A_1)\\
			&=&\frac{P(B_5)\cdot P(A_4)}{P(A_4)}P(A_4)+P(A_1)\\
			&=&P(B_5)P(A4)+P(A_1)\\
			&=&\frac{1}{2}\cdot \frac{b}{6}+ \frac{6-b}{b}\\
			&=&\frac{12-b}{12}
		\end{eqnarray*}
		Dieser Ausdruck ist nur dann größer als $\frac{1}{2}$, wenn $b<6$.
		Daraus schließen wir, dass mindestens eine Seite von Würfel 3 die Augenzahl 1 haben muss, damit der Ausdruck nach dem letzten Gleichheitszeichen größer ist als $\frac{1}{2}$.\\
		\medspace
		Aus den 3 Szenarien können wir nun schließen, dass es zwei Konfigurationen von Würfel 3 gibt, sodass das Spiel vorteilhaft für Spieler B ist. Spieler B kann also immer einen vorteilhaften Würfel wählen, wenn Würfel 3 vier Seiten hat mit einer Augenzahl 4 und 2 Seiten mit einer Augenzahl 1 hat, oder wenn Würfel 3 fünf Seiten mit einer Augenzahl 4 und eine Seite mit der Augnzahl 1 hat.
\end{Answer}


\begin{Exercise}
	In dieser Aufgabe betrachten wir eine bestimmte Anzahl an Familien in der Grundgesamtheit. Gegeben seien folgende Ereignisse für $i\in\left\lbrace 0,1,\dots 5\right\rbrace$:
	\begin{equation*}
		F_i:=\left\lbrace \text{Familie hat } i \text{ Kinder}\right\rbrace
	\end{equation*} 
	Hier gilt für eine zufällig ausgewählte Familie:
	\begin{eqnarray*}
		P(F_0)&=&0.3,\\
		P(F_1)&=&0.2,\\
		P(F_2)&=&0.2,\\
		P(F_3)&=&0.15,\\
		P(F_4)&=&0.1,\\
		P(F_5)&=&0.05.
	\end{eqnarray*}\\
	Wie groß ist die Wahrscheinlichkeit, dass ein aus diesen Familien zufällig ausgewählter Junge mindestens eine Schwester hat? Gehen Sie bei dieser Aufgabe, davon aus, dass die Wahrscheinlichkeiten für Jungen- und Mädchengeburt gleichgroß sind.
\end{Exercise}
\begin{Answer}
	Zunächst bestimmen wir die Wahrscheinlichkeit, dass ein Junge aus einer Familie mit $k\in\left\lbrace 0,1,\dots, 5\right.\rbrace$ Kindern kommt. Wir bezeichnen dieses Ereignis mit 
	\begin{equation*}
		J_k:=\left\lbrace \text{Zufällig ausgewählter Junge stammt aus Familie mit } k \text{ Kindern}\right\rbrace
	\end{equation*}
	Nimmt man an, dass es insgesamt $n\in\mathbb{N}$ Familien gibt, so gibt es insgesamt $\sum_{i=0}^5 k \cdot P(F_k)\cdot n=n\cdot\sum_{i=0}^5 k \cdot P(F_k)=n\cdot 1.7$ Kindern. Es gibt des Weiteren insgesamt $k\cdot P(F_k)\cdot n$ Kinder, die aus einer Familie mit $k$ Kindern stammen. Da es sich hier um eine Laplace Wahrscheinlichkeit handelt, gilt somit: $P(J_k)=\frac{k\cdot P(F_k)\cdot n}{n\cdot 1.7}=\frac{k\cdot P(F_k)}{1.7}$.\\
	Sei nun 
	\begin{equation*}
		A:=\left\lbrace\text{ Zufällig ausgewählter Junge hat mindestens eine Schwester}\right\rbrace.
	\end{equation*}
	Dann gilt
	\begin{equation*}
		\overline{A}:=\left\lbrace \text{Zufällig ausgewählter Junge hat nur Brüder oder kein Geschwisterkind}\right\rbrace.
	\end{equation*}
	Wir beobachten nun, dass $P(A\vert J_0)=0$, da ein Junge nicht aus einer Familie mit 0 Kindern stammen kann. Für $k>0$ gilt dann:
	\begin{eqnarray*}
		P(A\vert J_k)&=&1-P(\overline{A}\vert J_k)\\
		&=&1-P(\left\lbrace \text{Zufällig ausgewählter Junge hat } k-1 \text{ Brüder}\right\rbrace)\\
		&=& 1-\left(\frac{1}{2}\right)^{k-1}.
	\end{eqnarray*}
	Mit dem Satz der totalen Wahrscheinlichkeit ergibt sich nun:
	\begin{eqnarray*}
		P(A)&:=&\sum_{k=1}^5P(A\vert J_k)\cdot P(J_k)\\
		&=&\sum_{k=1}^5P(A\vert J_k)\cdot \frac{k\cdot P(F_k)}{1.7}\\
		&=&\frac{1}{1.7}\cdot \left(\left(1-\left(\frac{1}{2}\right)^0\right)\cdot 1 \cdot 0.2+\left(1-\left(\frac{1}{2}\right)^1\right)\cdot 2 \cdot 0.2+\left(1-\left(\frac{1}{2}\right)^2\right)\cdot 3 \cdot 0.15\right.\\
		&&\left. +\left(1-\left(\frac{1}{2}\right)^3\right)\cdot 4 \cdot 0.1+\left(1-\left(\frac{1}{2}\right)^4\right)\cdot 5 \cdot 0.05\right)\\
		&\approx&0.6667
	\end{eqnarray*}
\end{Answer}
\end{document}
