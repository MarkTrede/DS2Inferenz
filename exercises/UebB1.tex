\section[Grundlagen]{Grundlagen}

\begin{enumerate}
\item Eine M\"{u}nze wird zweimal geworfen. Erscheint mindestens einmal
"`Zahl"', so wird sie ein drittes Mal geworfen.
\begin{enumerate}
\item Geben Sie die Ergebnismenge an.
\item Wie viele Elemente hat das Ereignis "`Die M\"{u}nze wird dreimal
geworfen"'?
\end{enumerate}

\item Ein W\"{u}rfel wird einmal geworfen. Falls die Augenzahl
"`Sechs"' ist, wird der W\"{u}rfel noch einmal geworfen.
Wird wieder eine "`Sechs"' gew\"{u}rfelt, wird nochmals gew\"{u}rfelt usw.
\begin{enumerate}
\item Geben Sie die Ergebnismenge an.
\item Wie sieht das Ereignis "`Die ersten beiden Augenzahlen sind beide
Sechs"' aus?
\end{enumerate}

\item Ein W\"{u}rfel wird solange geworfen, bis die Summe aller Augenzahlen
mindestens vier ist.
\begin{enumerate}
\item Geben Sie die Ergebnismenge an.
\item Wie viele Elemente hat das Ereignis "`Die erste Augenzahl ist eine Eins"'?
\end{enumerate}

\item Fassen Sie in eigenen Worten kurz zusammen, was mit dem Begriff
"`Wahrscheinlichkeit"' 
\begin{enumerate}
\item in der Umgangssprache und
\item in der Wahrscheinlichkeitstheorie gemeint ist.
\end{enumerate}
\end{enumerate}
