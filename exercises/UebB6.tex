\section[Stetige Verteilungen]{Spezielle stetige Verteilungen}

\begin{enumerate}
\item Sei $X$ eine normalverteilte Zufallsvariable mit
Erwartungswert 20 und Standardabweichung 10. 
\begin{enumerate}
\item Wie gro\ss\ ist die Wahrscheinlichkeit, dass $X$ einen negativen Wert
annimmt?
\item Wie gro\ss\ ist die Wahrscheinlichkeit, dass $X$ einen Wert zwischen
15 und 25 annimmt?
\item Wie gro\ss\ ist die Wahrscheinlichkeit, dass $X$ einen Wert von
mindestens 22 annimmt?
\item Berechnen Sie das 0.1-Quantil, das 0.5-Quantil und das 0.975-Quantil
von $X$.
\end{enumerate}

\item Ein Hausratversicherer wei\ss\ aus Erfahrung, dass die Schadenh%
\"{o}he in 30.5\% aller Schadenf\"{a}lle h\"{o}chstens 1100 DM und in 16.6\%
mehr als 2000 DM betr\"{a}gt. Die Schadenh\"{o}he pro Schadenfall sei
normalverteilt.
\begin{enumerate}
\item Berechnen Sie den Mittelwert und die Standardabweichung f\"{u}r die
Schadenh\"{o}he pro Schadenfall.
\item Geben Sie f\"{u}r die Schadenh\"{o}he pro Schadenfall das 0.2-Quantil
an.
\item Wie gro\ss\ ist die Wahrscheinlichkeit daf\"{u}r, dass die Schadenh%
\"{o}he pro Schadenfall genau 1500 DM betr\"{a}gt?
\end{enumerate}

\item Die Zufallsvariable $X$ beschreibe die Lebensdauer eines
Monitors (in Stunden). Gehen Sie davon aus, dass $X$
exponentialverteilt ist mit dem Parameter $\lambda =0.00025$.
\begin{enumerate}
\item Bestimmen Sie den Erwartungswert von $X.$
\item Bestimmen Sie den Median von $X$.
\item Erstellen Sie eine Grafik der Dichtefunktion von $X$.
\item Der Verk\"{a}ufer gibt eine Garantie, dass der Monitor in den ersten
1000 Stunden nicht kaputt geht. Wie hoch ist die Wahrscheinlichkeit, dass
ein verkauftes Ger\"{a}t die Garantie nicht erf\"{u}llt?
\end{enumerate}

\item Value-at-Risk: @@@ ACHTUNG Diese Aufgabe muss \"{u}%
berarbeitet werden @@@

Sei $X$ die Tagesrendite (in \%) der Aktie der Deutschen Bank. Mit $Y=-X$
bezeichnen wir den prozentualen Verlust der Aktie an einem B\"{o}rsentag.
Die obere Flanke von $Y$ l\"{a}sst sich sehr gut durch eine Paretoverteilung
mit den Parametern $\alpha =3.4$ und $c=0.8$ modellieren, die
Verteilungsfunktion von $Y$ ist%
\begin{equation*}
F_{Y}(y)=1-\left( \frac{c}{y}\right) ^{\alpha }.
\end{equation*}
\begin{enumerate}
\item Wie gro\ss\ ist die Wahrscheinlichkeit, dass die Aktie der Deutschen
Bank morgen um mehr als 5\% an Wert verliert?
\item Wie gro\ss\ ist die Wahrscheinlichkeit, dass die Aktie der Deutschen
Bank morgen zwischen 5\% und 7\% an Wert verliert?
\item Bestimmen Sie das 0.99-Quantil von $Y$. Was sagt es aus? Solche
Quantile werden im Finance-Bereich auch als Value-at-Risk bezeichnet.
\item Angenommen, die Aktie f\"{a}llt morgen um mehr als 5\%. Wie gro\ss\ %
ist die Wahrscheinlichkeit, dass\ sie sogar um mehr als 7\% f\"{a}llt?
\end{enumerate}

\item Anreize oder: Wie eine erfolgsabh\"{a}ngige
Manager-Entlohnung schiefgehen kann.

Der Manager eines Unternehmens kann zwei Strategien einschlagen: eine
riskante und eine sichere. Die Zufallsvariable $X\sim N\left( \mu ,\sigma
^{2}\right) $ ist der Gewinn des Unternehmens (in Mio. EUR). Die Verteilung
von $X$ h\"{a}ngt von der gew\"{a}hlten Strategie ab, und zwar wie folgt:%
\begin{equation*}
\begin{array}{lll}
\text{sichere Strategie:} & \mu _{s}=6, & \sigma _{s}^{2}=9 \\ 
\text{riskante Strategie:} & \mu _{r}=5, & \sigma _{r}^{2}=25%
\end{array}%
\end{equation*}%
Das Gehalt $Y$ (in 10\thinspace 000 EUR) des Managers ist erfolgsabh\"{a}%
ngig (abh\"{a}ngig vom Gewinn). Es betr\"{a}gt%
\begin{equation*}
Y=\left\{ 
\begin{array}{rl}
3 & \quad \text{wenn }X\leq 0 \\ 
10 & \quad \text{wenn }0<X\leq 10 \\ 
25 & \quad \text{wenn }X>10.%
\end{array}%
\right.
\end{equation*}

\begin{enumerate}
\item Erg\"{a}nzen Sie die folgende Tabelle:

%TCIMACRO{\TeXButton{BC}{\begin{center}}}%
%BeginExpansion
\begin{center}%
%EndExpansion
\begin{tabular}{|l|c|c|}
\hline
& \multicolumn{2}{|c|}{Strategie} \\ \cline{2-3}
\rule{2cm}{0cm} & sicher & riskant \\ \hline
\multicolumn{1}{|c|}{$y$} & \multicolumn{1}{|l|}{$P\left( Y=y\right) $} & 
\multicolumn{1}{|l|}{$P\left( Y=y\right) $} \\ \hline
\rule{0cm}{3cm} & \multicolumn{1}{|l|}{} & \multicolumn{1}{|l|}{} \\ \hline
\end{tabular}%
%TCIMACRO{\TeXButton{EC}{\end{center}}}%
%BeginExpansion
\end{center}%
%EndExpansion
\medskip

\item Berechnen Sie den Erwartungswert des Manager-Gehalts $Y$ f\"{u}r beide
Strategien. Welche Strategie wird der Manager einschlagen, wenn er den
Erwartungswert seines Einkommens maximieren will?
\end{enumerate}

\item Der Preis der Streuung oder: Vom Vorteil sicherer Umwege.

Die Reisedauer mit dem Auto von A nach B h\"{a}ngt von vielen Zufallseinfl%
\"{u}ssen ab. Gehen Sie davon aus, dass die Reisedauer $X$ (in Std.) eine
normalverteilte Zufallsvariable ist mit Erwartungswert $\mu =6$ und einer
Standardabweichung von $\sigma =1$.
\begin{enumerate}
\item Wie gro\ss\ ist die Wahrscheinlichkeit, dass die Reise weniger als 3
Stunden dauert?
\item Sie sind in A. Um 16 Uhr haben Sie einen wichtigen Termin in B. Wann m%
\"{u}ssen Sie losfahren, damit Sie mit einer Wahrscheinlichkeit von 0.9 p%
\"{u}nktlich sind?
\item Wenn Sie einen Umweg \"{u}ber eine Autobahn fahren, verl\"{a}ngert
sich die erwartete Reisedauer auf $\mu =6.5$, aber wegen der besser
ausgebauten Stra\ss e verringert sich die Standardabweichung der Reisedauer
auf $\sigma =0.1$. Sie sind wieder in A und haben wieder um 16 Uhr einen
wichtigen Termin in B. Wann m\"{u}ssen Sie losfahren, damit Sie mit einer
Wahrscheinlichkeit von 0.9 p\"{u}nktlich sind?
\end{enumerate}

\item Sei $X\sim N(\mu ,\sigma ^{2})$. Definieren Sie die
Zufallsvariable $Y=e^{X}$. Man nennt $Y$ lognormalverteilt mit den
Parametern $\mu $ und $\sigma ^{2}$ und schreibt $Y\sim LN(\mu ,\sigma
^{2}). $ Dies ist das einfachste Standardmodell f\"{u}r die Verteilung zuk%
\"{u}nftiger Aktienkurse. Es bildet unter anderem eine wichtige Grundlage f%
\"{u}r die Optionsbewertung. Zeigen Sie, dass gilt%
\begin{equation*}
E(Y)=e^{\mu +\sigma ^{2}/2}.
\end{equation*}%
Hinweis: Diese Aufgabe ist schwierig. Wegen $E(Y)=E(e^{X})$ k\"{o}nnen Sie
den Erwartungswert \"{u}ber die Formel $E(g(X))=\int_{-\infty }^{\infty
}g(x)f(x)dx$ berechnen. Fassen Sie dann alle Exponentialfunktionsausdr\"{u}%
cke geeignet zusammen und nutzen Sie die Tatsache, dass f\"{u}r jede Dichte $%
\int_{-\infty }^{\infty }f(x)dx=1$ gilt.

\item Aktienoptionen:

Die Verteilung von Aktienkursen, die etwas weiter entfernt in der Zukunft
liegen (z.B. einen Monat oder l\"{a}nger), l\"{a}sst sich recht gut durch
eine Lognormalverteilung beschreiben. Sei $S_{0}=100$ der heutige Kurs einer
Aktie. Der Kurs in einem Jahr, $S_{1}$, ist nat\"{u}rlich noch unbekannt und
wird daher als Zufallsvariable aufgefasst. Wir nehmen an, dass $S_{1}$
lognormalverteilt ist: $S_{1}\sim LN(\mu ,\sigma ^{2})$ mit den
(vorgegebenen) Parametern $\mu =4.685$ und $\sigma ^{2}=0.2$.
\begin{enumerate}
\item Eine Kauf-Option bietet das Recht, zu einem vereinbarten zuk\"{u}%
nftigen Zeitpunkt eine Aktie zu einem vereinbarten Kurs (Aus\"{u}bungspreis)
zu kaufen -- unabh\"{a}ngig vom dann geltenden Aktienkurs. Nach der ber\"{u}%
hmten Formel von Black und Scholes kann der heutige Wert $C_{0}$ einer
Kauf-Option bestimmt werden als%
\begin{eqnarray*}
C_{0} &=&S_{0}\Phi \left( \frac{\ln \left( S_{0}/K\right) +\left( \rho
+\sigma ^{2}/2\right) T}{\sigma \sqrt{T}}\right) \\
&&-e^{-\rho T}K\Phi \left( \frac{\ln \left( S_{0}/K\right) +\left( \rho
-\sigma ^{2}/2\right) T}{\sigma \sqrt{T}}\right) ,
\end{eqnarray*}%
wobei $S_{0}$ der aktuelle Aktienkurs ist, $K$ der Aus\"{u}bungspreis, $\rho 
$ der sichere (und als konstant angenommene) Zinssatz, $\sigma ^{2}$ die
Varianz der Rendite (das entspricht in unserem hier behandelten Fall dem
Parameter $\sigma ^{2}$ der Lognormalverteilung) und $T$ die Laufzeit der
Option.

Berechnen Sie den heutigen Wert der Option $C_{0}$ f\"{u}r einen Aus\"{u}%
bungspreis von $K=105$, einen Zinssatz von 5\% p.a. und eine Restlaufzeit
von $T=1$ (Jahr).

\item In einem Jahr ist der Wert der Option, $C_{1}$, nat\"{u}rlich abh\"{a}%
ngig vom dann geltenden Aktienkurs. Daher k\"{o}nnen wir $C_{1}$ ebenso wie $%
S_{1}$ als Zufallsvariable auffassen. Wenn $S_{1}\leq K$ ist, dann ist die
Option offenbar wertlos (d.h. $C_{1}=0$). Wenn hingegen $S_{1}>K$ ist, dann
ist der Wert der Option gerade $C_{1}=S_{1}-K$.

Bestimmen Sie die Wahrscheinlichkeit, dass $C_{1}=0$ ist.

\item Bestimmen Sie den Median von $S_{1}$.

\item Bestimmen Sie den Median von $C_{1}$.
\end{enumerate}

\item Weibull-Verteilung oder: Abnutzungserscheinungen.

Betrachten Sie wieder die Zufallsvariable $X$ aus Aufgabe 3: $X$ beschreibt
die Lebensdauer eines Fernsehger\"{a}ts (in Stunden). Gehen Sie davon aus,
dass $X$ exponentialverteilt ist mit dem Parameter $\lambda =0.00025$.

\begin{enumerate}
\item Zeigen Sie: Die Wahrscheinlichkeit, dass der Fernseher mehr als 6000
Stunden h\"{a}lt, wenn er bereits 4000 Stunden gehalten hat, ist genauso gro%
\ss\ wie die Wahrscheinlichkeit, dass der Fernseher mehr als 10000 Stunden h%
\"{a}lt, wenn er bereits 8000 Stunden gehalten hat.
\item Zeigen Sie allgemein: $P(X>a|X>b)$ mit $a,b\in \mathbb{R}$ und $a>b$ h%
\"{a}ngt bei einer Exponentialverteilung nur von der Differenz $a-b$ ab.
\item Um eine mit dem Alter steigende Ausfallwahrscheinlichkeit zu
modellieren, verwendet man h\"{a}ufig die Weibull-Verteilung. Eine
Zufallsvariable $Y$ hei\ss t Weibull-verteilt mit den Parametern $\beta >0$
und $k>0$, wenn ihre Verteilungsfunktion f\"{u}r $y\geq 0$%
\begin{equation*}
F_{Y}(y)=1-e^{-(y/\beta )^{k}}
\end{equation*}%
lautet. F\"{u}r $y<0$ ist $F_{Y}(y)=0$.
\item Bestimmen Sie die Formel f\"{u}r die Dichtefunktion von $Y$.
\item Die Lebensdauer des Fernsehger\"{a}ts (in Stunden) beschreiben wir nun
durch eine Weibull-verteilte Zufallsvariable $Y$ mit den Parametern $\beta
=4431$ und $k=1.5.$ Zeichnen Sie die Dichtefunktion von $Y$ (z.B. mit dem
Statistik-Programm R).
\item Berechnen und vergleichen Sie nun die beiden folgenden bedingten
Wahrscheinlichkeiten: Die Wahrscheinlichkeit, dass der Fernseher mehr als
6000 Stunden h\"{a}lt, wenn er bereits 4000 Stunden gehalten hat. Die
Wahrscheinlichkeit, dass der Fernseher mehr als 10000 Stunden h\"{a}lt, wenn
er bereits 8000 Stunden gehalten hat.
\end{enumerate}
\end{enumerate}
