\section[Zufallsvariablen]{Zufallsvariablen, Erwartungswert, Varianz}

\begin{enumerate}
\item Sei $X$ eine diskrete Zufallsvariable mit Tr\"{a}ger $T_{X}=\left\{
1,2,3,4,5,6\right\} $ und der Wahrscheinlichkeitsfunktion 
\begin{equation*}
P\left( X=x\right) =\left\{ 
\begin{array}{ll}
0.1 & \quad \text{f\"{u}r }x=1,2,3,4,5 \\ 
0.5 & \quad \text{f\"{u}r }x=6.%
\end{array}
\right.
\end{equation*}
\begin{enumerate}
\item Skizzieren Sie die Verteilungsfunktion von $X$.
\item Skizzieren Sie die Verteilungsfunktion von $Y=X^{2}$.
\item Berechnen Sie den Erwartungswert von $Y$.
\item Berechnen Sie die Varianz von $Y$.
\end{enumerate}

\item Die Dichte einer stetigen Zufallsvariablen $X$ ist 
\begin{equation*}
f(x)=\left\{ 
\begin{array}{ll}
x+0.5 & \qquad \text{f\"{u}r }0<x<1 \\ 
0 & \qquad \text{sonst.}%
\end{array}%
\right.
\end{equation*}%
Errechnen Sie den Erwartungswert, den Median und die Varianz von $X$.
\item Bestimmen Sie $c$ so, dass 
\begin{equation*}
f(x)=\left\{ 
\begin{array}{ll}
cx^{2} & \qquad \text{f\"{u}r }0<x<2 \\ 
0 & \qquad \text{sonst}%
\end{array}%
\right.
\end{equation*}
eine Dichtefunktion wird. Die stetige Zufallsvariable $X$ habe die oben
angegebene Dichte.
\begin{enumerate}
\item Berechnen Sie den Erwartungswert von $X$.
\item Berechnen Sie das 10\%-Quantil von $X$.
\end{enumerate}

\item Von der Verteilungsfunktion $F$ einer Zufallsvariablen $X$ ist
bekannt, dass $F(x)=x^{2}$ f\"{u}r $0<x<1$ gilt.
\begin{enumerate}
\item Welche Werte muss $F$ f\"{u}r $x<0$, $x=0$, $x=1$, $x>1$ annehmen?
Skizzieren Sie die gesamte Verteilungsfunktion $F$.
\item Berechnen Sie die Dichtefunktion von $X$.
\item Errechnen Sie $E(X)$.
\item Errechnen Sie $Var(X)$.
\item Errechnen Sie den Median von $X$.
\item Bestimmen Sie $P(X=0.5)$.
\end{enumerate}

\item Bei einer Tombola enth\"{a}lt eine Lostrommel 2000 Lose, von denen
80\% Nieten und 20\% Gewinnlose sind. 10\% der Gewinnlose bringen eine
Auszahlung von jeweils 100 Euro, die Auszahlung der \"{u}brigen 90\% der
Gewinnlose betr\"{a}gt jeweils 10 Euro.

Wie hoch sind Erwartungswert und die Varianz des Gewinns beim Kauf eines
Loses, wenn ein Los 5 Euro kostet?

\item Ein Kunde einer Bank will einen Kredit haben. Der Kredit soll
100\thinspace 000 Euro betragen, \"{u}ber ein Jahr laufen und am Ende des
Jahres getilgt werden. Die Bank stellt Nachforschungen \"{u}ber die Bonit%
\"{a}t des Kunden an und erf\"{a}hrt, dass das Kreditausfallrisiko bei 10\%
liegt (d.h. die Wahrscheinlichkeit, dass der Kunde weder Zinsen noch Tilgung
zahlt, ist 10\%; mit einer Wahrscheinlichkeit von 90\% wird korrekt zur\"{u}%
ckgezahlt).

Welchen Zinssatz muss die Bank fordern, damit sie eine erwartete Rendite von
6\% erzielt? Wie hoch ist bei diesem Zinssatz die Varianz der Rendite?

\item Ein Kunde einer Bank will einen Kredit haben. Der Kredit soll 1 Mio.\
Euro betragen, \"{u}ber ein Jahr laufen und am Ende des Jahres getilgt
werden. Die Bank stellt Nachforschungen \"{u}ber die Bonit\"{a}t des Kunden
an und erf\"{a}hrt,

\begin{itemize}
\item dass der Kunde mit einer Wahrscheinlichkeit von 70\% Zinsen und Tilgung zahlt,
\item dass der Kunde mit einer Wahrscheinlichkeit von 10\% weder Zinsen noch Tilgung zahlt,
\item dass der Kunde mit einer Wahrscheinlichkeit von 20\% zwar keine Zinsen
zahlt, aber einen Betrag von $X$ tilgt, wobei $X$ eine stetige
Zufallsvariable ist (in Mio.\ Euro) mit Dichtefunktion%
\begin{equation*}
f_{X}\left( x\right) =\left\{ 
\begin{array}{ll}
0 & \quad \text{f\"{u}r }x<0 \\ 
2x & \quad \text{f\"{u}r }0\leq x\leq 1 \\ 
0 & \quad \text{f\"{u}r }x>1.%
\end{array}%
\right.
\end{equation*}
\end{itemize}

Welchen Zinssatz m\"{u}sste die Bank fordern, damit sie eine erwartete
Rendite von 6\% erzielt?

\item Petersburger Paradoxon:\footnote{%
Dieses Paradox wurde von Daniel Bernoulli in dem Aufsatz \quotedblbase
Specimen Theoriae Novae de Mensura Sortis\textquotedblright , Commentarii
Academiae Scientiarum Imperialis Petropolitanae, Tomus V [Papers of the
Imperial Academy of Sciences in Petersburg, Vol. V], 1738, pp. 175-192,
behandelt. Eine englische \"{U}bersetzung der Arbeit erschien 1954 in:
Econometrica, vol. 22, pp. 23-36.}

Ein Spiel hei\ss t fair, wenn der erwartete Gewinn 0 ist. Betrachten Sie
folgendes Spiel: Eine M\"{u}nze wird so oft geworfen, bis zum ersten Mal
Zahl erscheint. Wenn schon beim ersten Wurf Zahl erscheint, werden 2 Euro
ausgezahlt. Wenn beim zweiten Wurf Zahl erscheint, werden 4 Euro ausgezahlt.
Wenn beim dritten Wurf Zahl erscheint, werden 8 Euro ausgezahlt. Allgemein
werden $2^{i}$ Euro ausgezahlt, wenn beim $i$-ten Wurf zum ersten Mal Zahl
erscheint.
\begin{enumerate}
\item Wie hoch m\"{u}sste der Einsatz sein, damit das Spiel fair ist?
\item Bernoulli hat argumentiert, dass nicht die Auszahlung eines Spiels,
sondern der Nutzen der Auszahlung entscheidend ist. Als Nutzen $u$ einer
Auszahlung von $x$ Euro nahm Bernoulli $u\left( x\right) =\ln \left(
x\right) $ an (der Grenznutzen ist also fallend). Wieviel ist jemand mit
dieser Nutzenfunktion bereit, f\"{u}r das Spiel zu zahlen?

Hinweis: Berechnen Sie zun\"{a}chst den Erwartungswert des Nutzens des
Gewinns; dabei k\"{o}nnen Sie die Gleichheit $\sum_{i=1}^{\infty }i/2^{i}=2$
ausnutzen. Ein Wirtschaftssubjekt zahlt f\"{u}r das Spiel maximal einen
Einsatz, dessen Nutzen dem Erwartungswert des Nutzens des Gewinns entspricht.
\end{enumerate}

\item Customer Lifetime Value oder Kundenwertrechnung:

Als Lebensdeckungsbeitrag bezeichnet man die Summe aller diskontierten j\"{a}%
hrlichen Deckungsbeitr\"{a}ge \"{u}ber die gesamte Dauer einer
Kundenbeziehung. Die folgende Tabelle zeigt die K\"{u}%
ndigungswahrscheinlichkeiten und die j\"{a}hrlichen Deckungsbeitr\"{a}ge
eines Kunden. Die K\"{u}ndigungswahrscheinlichkeiten geben an, wie gro\ss\ %
die Wahrscheinlichkeit ist, dass der Kunde nach genau x Jahren k\"{u}ndigt.
Vernachl\"{a}ssigen Sie im Folgenden die Abdiskontierung (der Zinssatz sei
0\%).
\begin{center}
\begin{tabular}{lcc}
\hline
Dauer & K\"{u}ndigungs- & j\"{a}hrlicher \\ 
& wahrscheinlichk. & Deckungsbeitr. \\ \hline
1 Jahr & 0.5 & $-5$ \\ 
2 Jahre & 0.0 & $+8$ \\ 
3 Jahre & 0.5 & $-5$ \\ \hline
\end{tabular}
\end{center}
\begin{enumerate}
\item Berechnen Sie die erwartete Dauer der Kundenbeziehung.
\item Berechnen Sie den Lebensdeckungsbeitrag des Kunden, wenn die Dauer der
Kundenbeziehung gerade der erwarteten Dauer der Kundenbeziehung entspricht.
\item Berechnen Sie den erwarteten Lebensdeckungsbeitrag.
\end{enumerate}

\item Sinnloses Jugend-Marketing:

Eine Bank will Jugend-Marketing betreiben. Der Datensatz \texttt{jugend.csv}
zeigt die einj\"{a}hrigen Deckungsbeitr\"{a}ge von Bankkunden sowie ihre einj%
\"{a}hrigen K\"{u}ndigungswahrscheinlichkeiten\footnote{%
Unter der einj\"{a}hrigen K\"{u}ndigungswahrscheinlichkeit versteht man die
bedingte Wahrscheinlichkeit, dass jemand innerhalb eines Jahres k\"{u}ndigt,
der bislang noch nicht gek\"{u}ndigt hat.} in Abh\"{a}ngigkeit vom Alter.
\begin{enumerate}
\item Die gro\ss\ ist die Wahrscheinlichkeit, dass ein 15-j\"{a}hriger
Neukunde im Alter von 35 Jahren immer noch Kunde der Bank ist?
\item Der Diskontierungssatz sei 4\%. Wie gro\ss\ ist der erwartete
Lebensdeckungsbeitrag eines 15-j\"{a}hrigen Neukunden? Lohnt sich unter
diesen Umst\"{a}nden das Jugendmarketing?
\item Wie gro\ss\ ist der erwartete Lebensdeckungsbeitrag eines 40-j\"{a}%
hrigen Neukunden?
\end{enumerate}

\item Insiderhandel oder: Warum es schlau sein kann zu gucken, was die anderen tun.

Stellen Sie sich vor, Sie seien Market Maker an einer Wertpapierb\"{o}rse
und daher verpflichtet, Liquidit\"{a}t bereitzustellen, d.h. einem H\"{a}%
ndler Aktien zu verkaufen, wenn er Aktien kaufen m\"{o}chte, und einem H\"{a}%
ndler Aktien abzukaufen, wenn er Aktien verkaufen m\"{o}chte.

Ihnen ist bekannt, dass 10\% aller Marktteilnehmer Insider sind. Insider
kennen den zu\-k\"{u}nf\-ti\-gen Aktienkurs $K_{1}$, sie werden daher nur
handeln, wenn sie ihren Informationsvorteil kapitalisieren k\"{o}nnen. Die
uninformierten H\"{a}ndler (die oft auch Noise-Trader genannt werden)
hingegen kaufen und verkaufen unabh\"{a}ngig vom Aktienkurs mit einer
Wahrscheinlichkeit von jeweils 50\%. Die H\"{a}ndler d\"{u}rfen nur
begrenzte Mengen handeln. Leider k\"{o}nnen Sie die Insider nicht von den
uninformierten H\"{a}ndlern unterscheiden.

Den zuk\"{u}nftigen Aktienkurs kennen Sie als Market Maker nicht. F\"{u}r
Sie ist $K_{1}$ eine Zufallsvariable, deren Verteilung zur Vereinfachung 
\begin{equation*}
P(K_{1}=95)=P(K_{1}=105)=0.5
\end{equation*}%
sei. Der aktuelle Aktienkurs, zu dem Sie als Market Maker kaufen und
verkaufen, sei $K_{0}=100$. Die n\"{a}chste Order, die Sie erhalten, ist
eine Kauf-Order.

\begin{enumerate}
\item Revidieren Sie Ihre Wahrscheinlichkeitsverteilung von $K_{1}$ unter Ber%
\"{u}cksichtigung der Tatsache, dass Sie eine Kauf-Order (und keine
Verkauf-Order) erhalten haben. Hinweise: Definieren Sie geeignete Ereignisse
und nutzen Sie den Satz von Bayes sowie den Satz der totalen
Wahrscheinlichkeit. Strukturieren Sie Ihre L\"{o}sung mit Hilfe eines
Wahrscheinlichkeitsbaums.
\item Welchen Erwartungswert hat (aus Ihrer Sicht als Market Maker) die
Zufallsvariable $K_{1}$ nach Eingang der Kauf-Order?
\end{enumerate}

\item Unterschiedliche Meinungen oder: Wenn zwei sich nicht einig sind,
sollten sie wetten.

Wahrscheinlichkeiten k\"{o}nnen subjektive Einsch\"{a}tzungen repr\"{a}%
sentieren. Betrachten Sie im folgenden zwei Personen, die die
Wahrscheinlichkeit des Eintretens eines Ereignisses $A$ unterschiedlich
einsch\"{a}tzen. (Um es konkret zu machen: sei $A$ das Ereignis
\quotedblbase In einem Monat steht der Dollar h\"{o}her als
heute\textquotedblright .) Person 1 glaubt, dass $P_{1}(A)=0.9$ betr\"{a}gt.
Person 2 glaubt, dass $P_{2}(A)=0.3$.

Beide Personen haben identische Nutzenfunktionen $u(y)=-1/y$, die den Nutzen
eines Einkommens in H\"{o}he von $y$ angeben.\footnote{%
Falls Sie in Mikro\"{o}konomik die Erwartungsnutzentheorie noch nicht
behandelt haben, nehmen Sie die Existenz einer solchen Nutzenfunktion
einfach als gegeben hin. Dass negative Nutzenwerte vorkommen, ist
unproblematisch. Wichtig ist nur, dass h\"{o}here Nutzenwerte besser sind
als niedrigere.} Beide Personen haben ein sicheres Einkommen von $y_{0}=5$.
Da die Personen unterschiedlicher Meinung sind, macht es Sinn, eine Wette zu
vereinbaren. Sei $X$ der Betrag $b>0$, den Person 2 an Person 1 zahlt, wenn $%
A$ eintritt. Falls $\bar{A}$ eintritt, empf\"{a}ngt Person 2 eine Zahlung
von Person 1 (d.h. $X$ ist negativ);%
\begin{equation*}
X=\left\{ 
\begin{array}{ll}
+b & \quad \text{wenn }A \\ 
-b & \quad \text{wenn }\bar{A}.%
\end{array}%
\right.
\end{equation*}%
Die Einkommen der beiden Personen sind nun nicht mehr sicher, sondern zuf%
\"{a}llig. Ebenso sind auch die beiden Nutzen Zufallsvariablen: der Nutzen
von Person 1 ist $u(y_{0}+X)$, und der Nutzen von Person 2 ist $u(y_{0}-X)$.
\begin{enumerate}
\item Berechnen Sie den erwarteten Nutzen der beiden Personen als Funktion
von $b$. Achten Sie auf die richtigen Wahrscheinlichkeiten bei der
Berechnung der Erwartungswerte.
\item Skizzieren Sie die beiden erwarteten Nutzen in Abh\"{a}ngigkeit vom
Wetteinsatz $b$.
\item Auf welchen Wetteinsatz werden sich die beiden Personen einigen?
\end{enumerate}

\item Dynamische Optimierung oder: Vom Wert des Wartens.

Diese Aufgabe basiert auf einem einf\"{u}hrenden Beispiel des Buches
\quotedblbase Investment under Uncertainty\textquotedblright\ von Dixit und
Pindyck, 1994, S. 27f. Sie sollen entscheiden, ob eine Fabrik zur
Herstellung eines bestimmten Produkts gebaut werden soll. Die Investition,
also der Bau der Fabrik, kostet 1600 Euro. Das Produkt hat heute (in $t=0$)
einen Preis von 200 Euro. In der n\"{a}chsten Periode ($t=1$) \"{a}ndert
sich jedoch der Preis des Produkts. Mit einer Wahrscheinlichkeit von 0.5
steigt er auf 300 Euro und mit einer Wahrscheinlichkeit von 0.5 sinkt er auf
100 Euro. Danach bleibt der Preis f\"{u}r alle Zeiten konstant (also f\"{u}r 
$t=2,3,\ldots $). Der Zinssatz sei 10\%. Die Fabrik kann jede Periode eine
Einheit des Produkts herstellen; die Herstellung verursacht keine Kosten.
Die Investition kann nicht wieder r\"{u}ckg\"{a}ngig gemacht werden, wenn
sie einmal durchgef\"{u}hrt wurde. (Die teilweise sehr unrealistischen
Annahmen dienen allein dazu, den Blick auf den wesentlichen Kern dieses
Beispiels zu lenken.)

\begin{enumerate}
\item Unter dem Nettobarwert (net present value) versteht man die Summe
aller abgezinsten erwarteten Kosten und Erl\"{o}se. Berechnen Sie den
Nettobarwert der Investition. Sollte die Investition durchgef\"{u}hrt werden?
\item Eine m\"{o}gliche Handlungsoption wurde unter (a) ignoriert: Man kann
die Entscheidung f\"{u}r oder gegen die Investition aufschieben. Berechnen
Sie den Nettobarwert (f\"{u}r $t=0$), wenn in $t=0$ abgewartet wird, wie
sich der Preis entwickelt, und wenn in $t=1$ nur dann investiert wird, falls
der Preis auf 300 Euro steigt. F\"{a}llt der Preis, wird nicht investiert.
\end{enumerate}

\item Jensens Ungleichung:

Sei $X$ eine Zufallsvariable und $g$ eine konvexe Funktion. Jensens
Ungleichung besagt, dass $E(g(X))\geq g(E(X))$ ist. Illustrieren Sie die G%
\"{u}ltigkeit der Ungleichung durch ein einfaches Beispiel mit einer
diskreten Zufallsvariable.
\end{enumerate}
