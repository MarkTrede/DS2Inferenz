\section[Bed.~Wahrscheinlichkeit]{Bedingte Wahrscheinlichkeit und Unabhängigkeit}

\begin{enumerate}
\item In einer Bank sind zwei unabh\"{a}ngig voneinander arbeitende
Geldautomaten aufgestellt. Es ist bekannt, dass w\"{a}hrend einer Woche die
Ausfallwahrscheinlichkeiten f\"{u}r die beiden Automaten 30\% bzw. 20\%
betragen. Wir gro\ss\ ist die Wahrscheinlichkeit, dass im Laufe einer Woche
\begin{enumerate}
\item mindestens ein Geldautomat ausf\"{a}llt?
\item beide Geldautomaten ausfallen?
\item kein Geldautomat ausf\"{a}llt?
\item genau ein Geldautomat ausf\"{a}llt?
\end{enumerate}

\item In einer Fabrik f\"{u}r Computer-Chips produzieren drei Maschinen A,
B, C. Maschine A produziert 25\%, B produziert 35\%, und C produziert 40\%
der Chips. Die Ausschussanteile sind 5\%, 4\% und 2\%. Ein Chip der
Gesamtproduktion wird zuf\"{a}llig ausgew\"{a}hlt. Er ist kaputt. Wie groß
ist die Wahrscheinlichkeit, dass er an Maschine A (B,C) produziert
wurde?

\item Ein Dienstleistungsunternehmen m\"{o}chte die Zufriedenheit seiner
Kunden untersuchen. Dazu werden an zuf\"{a}llig ausgew\"{a}hlte Kunden 
Fragebögen verschickt, auf denen die Kunden ankreuzen k\"{o}nnen, ob sie
zufrieden oder unzufrieden sind (mehr Antwortm\"{o}glichkeiten soll es zur
Vereinfachung nicht geben). Die Wahrscheinlichkeit, dass ein zuf\"{a}llig
ausgew\"{a}hlter Kunde zufrieden ist, betr\"{a}gt 70\%. Leider antworten
nicht alle Kunden auf den Fragebogen. Die Antwortwahrscheinlichkeit eines
angeschriebenen Kunden betr\"{a}gt 60\%, wenn der Kunde zufrieden ist, aber
nur 15\%, wenn der Kunde unzufrieden ist.
\begin{enumerate}
\item Wie gro\ss\ ist die Wahrscheinlichkeit, dass ein Kunde zufrieden ist,
wenn er auf die Fragebogenaktion antwortet? Was bedeutet diese
Wahrscheinlichkeit?
\item Obwohl nicht alle Kunden auf die Fragebogenaktion antworten, kann man
Grenzen angeben, in denen die Wahrscheinlichkeit liegen muss, dass ein Kunde
zufrieden ist. Bestimmen Sie diese Grenzen.
\end{enumerate}

\item Die Subprime-Mortgages-Krise oder: Die Grenzen der Diversifikation.

Im Jahr 2008 wurden die internationalen Finanzm\"{a}rkte durch
eine Krise auf dem Kreditmarkt ersch\"{u}ttert. Die deutsche IKB Bank hat in
Folge der Krise einen gro\ss en Verlust erlitten. Offenbar war ein Problem
die falsche Beurteilung der Risiken eines Kre\-dit\-portfolios, also eines
Bündels von Krediten. Durch B\"{u}ndelung von Einzelrisiken wird im
Allgemeinen eine Diversifikation erreicht, die das Gesamtrisiko senkt. Im
Folgenden wird ein stark vereinfachtes Modell eines Kreditportfolios
behandelt, an dem die Folgen m\"{o}glicher Fehleinsch\"{a}tzungen deutlich
gemacht werden k\"{o}nnen. Nutzen Sie R, um die Berechnungen durchzuführen.

Eine Bank hat 100 Kreditkunden $i=1,\ldots ,100$. Sei $A_{i}$ das Ereignis
"`Der Kredit des Kunden $i$ f\"{a}llt innerhalb eines Jahres aus"'.
Wir nehmen vereinfachend an, dass Kredite nur ganz
oder gar nicht ausfallen k\"{o}nnen. Die Wahrscheinlichkeit eines Ausfalls
ist $P(A_{i})=0.05$ (das ist ziemlich hoch, es handelt sich also um subprime
mortgages).

Ein wichtiges Instrument f\"{u}r das Management von Kreditportfolios sind
CDS (credit default swaps). Mit einer speziellen Version, den "`$k$-th to 
default swaps"', lassen sich Kreditportfolios
absichern. Derartige Swaps sind Versicherungen gegen den Ausfall von $k$
oder mehr Kreditnehmern in einem Kreditportfolio.

Gehen Sie davon aus, dass die Kreditausf\"{a}lle der Kunden $i=1,\ldots ,100$
stochastisch unabh\"{a}ngig voneinander sind. (Wie wir in der n\"{a}chsten
Aufgabe sehen werden, ist das eine kritische Annahme.)
\begin{enumerate}
\item Wie gro\ss\ ist die Wahrscheinlichkeit, dass kein einziger Ausfall
eintritt?
\item Wie gro\ss\ ist die Wahrscheinlichkeit, dass genau drei Ausf\"{a}lle
eintreten?
\item Wie gro\ss\ ist die Wahrscheinlichkeit, dass mehr als zehn Ausfälle 
eintreten? Mit dieser Wahrscheinlichkeit l\"{a}sst sich der Wert eines
"`$k$-th to default swaps"' (f\"{u}r $k=11$) ermitteln. Es ist daher 
wichtig, dass diese Wahrscheinlichkeit korrekt bestimmt wird.
\item Wie gro\ss\ w\"{a}re die Wahrscheinlichkeit, dass mehr als zehn 
Ausfälle eintreten, wenn $P(A_{i})=0.07$ ist (wenn also das Rating falsch
gewesen wäre)?
\end{enumerate}

\item Die Krise der Subprime Mortgages oder: Die Grenzen der Diversifikation
(Fortsetzung).

Tats\"{a}chlich sind Kreditausf\"{a}lle nicht unabh\"{a}ngig voneinander. 
Fällt ein Kredit aus -- insbesondere im Bereich der subprime mortgages
--, dann liegt das meist ein einem ung\"{u}nstigen gesamtwirtschaftlichen
Umfeld. Damit sind auch die \"{u}brigen Kredite anf\"{a}lliger f\"{u}r einen
Ausfall. Um diese Abh\"{a}ngigkeit zu modellieren, erweitern wir unser
Modell:

Sei $B$ das Ereignis "`Die Konjunktur ist gut (Boom)"' und $\bar{B}$ das 
Ereignis "`Die Konjunktur ist schlecht (Flaute)"'. Sei $P(B)=0.5$. Im Fall
eines Booms ist die Wahrscheinlichkeit eines Kreditausfalls gering, 
$P(A_{i}|B)=0.01$ f\"{u}r $i=1,\ldots ,100$. Im Fall einer Flaute gilt jedoch 
$P(A_{i}|\bar{B})=0.09$. Die Kreditausf\"{a}lle seien bedingt 
unabhängig, d.h. f\"{u}r $i\neq j$ ist
\[ P(A_{i}\cap A_{j}|B)=P(A_{i}|B)\cdot P(A_{j}|B) \]
und analog f\"{u}r $\bar{B}$.
\begin{enumerate}
\item Wie gro\ss\ ist die Wahrscheinlichkeit, dass der Kredit des Kunden $i$
innerhalb eines Jahres ausf\"{a}llt?
\item Zeigen Sie, dass $A_{i}$ und $A_{j}$ (f\"{u}r $i\neq j$) nun
stochastisch abh\"{a}ngig sind.
\item Wie gro\ss\ ist die Wahrscheinlichkeit, dass mehr als zehn Ausfälle 
eintreten, wenn die Wirtschaft boomt?
\item Wie gro\ss\ ist die Wahrscheinlichkeit, dass mehr als zehn 
Ausfälle eintreten, wenn eine Konjunkturflaute eintritt?
\item Wie gro\ss\ ist die Wahrscheinlichkeit, dass mehr als zehn 
Ausfälle eintreten? Vergleichen Sie das Ergebnis mit der unter 5.c) ermittelten
Wahrscheinlichkeit.
\end{enumerate}
Man sieht: F\"{u}r die Beurteilung des Risikos eines Kreditportfolios ist
nicht nur ein exaktes Rating aller Einzelkredite wichtig, sondern auch eine
exakte Erfassung der Abh\"{a}ngigkeit der Kreditausf\"{a}lle. Das Rating
eines Kreditportfolios ist daher ungleich schwieriger als das Rating eines
Einzelkredits.

\item Informationskaskaden oder: Wie sich eine Mode durchsetzt.

In dieser Aufgabe wird eine stark vereinfachte Version des Artikels
"`A Theory of Fads, Fashion, Custom, and Cultural Change as 
Informational Cascades"' von Sushil Bikhchandani, David
Hirshleifer und Ivo Welch im Journal of Political Economy, vol.~100, 
S.~992-1026, 1992, vorgestellt. In
dieser Aufgabe liegt die Schwierigkeit darin, dass die Wahrscheinlichkeiten
die subjektiven -- und m\"{o}glicherweise unterschiedlichen -- Einschätzungen 
einzelner Personen repr\"{a}sentieren.

Zwanzig Personen sollen sich nacheinander entscheiden, ob sie ein bestimmtes
Verhalten annehmen (eine bestimmte Handlung durchf\"{u}hren, eine bestimmte Mütze 
aufsetzen, einen bestimmten Studiengang belegen, eine bestimmte
medizinische Behandlung anwenden, \ldots ). Sei $A_{i}$ das Ereignis
"`Person $i$ nimmt das Verhalten an"', $i=1,\ldots ,20$.

Leider wei\ss\ keine der 20 Personen, ob es n\"{u}tzlich ist, das Verhalten
anzunehmen. Sei $B$ das Ereignis "`Das Verhalten ist nützlich"'. 
Alle Personen haben anfangs die gleiche "`a-priori"' 
Einsch\"{a}tzung, dass $P(B)=0.5$ ist.

Jede Person empf\"{a}ngt nun ein "`privates Signal"'\footnote{Privat 
bedeutet, dass jede Person nur ihr eigenes Signal kennt. Die
empfangenen Signale werden nicht kommuniziert.} \"{u}ber die Nützlichkeit 
des Verhaltens. Sei $S_{i}$ das Ereignis "`Person $i$ empf\"{a}ngt das 
Signal, dass das Verhalten n\"{u}tzlich ist"' und $\bar{S}_{i}$ das Ereignis 
"`Person $i $ empf\"{a}ngt das Signal, dass das Verhalten sch\"{a}dlich
ist"'. Es gelte $P(S_{i}|B)=0.6$ und $P(S_{i}|\bar{B})=0.4$. 
Die Signale $S_{1},\ldots ,S_{20}$ seien bedingt unabh\"{a}ngig
voneinander, d.h. $P(S_{i}\cap S_{j}|B)=P(S_{i}|B)\cdot P(S_{j}|B)$ f\"{u}r 
$i\neq j$ und analog f\"{u}r $\bar{B}$.

Bis auf die erste Person kann jede Person $i$ beobachten, was die
vorhergehenden Personen $1,\ldots ,i-1$ getan haben.

Person $i$ nimmt das Verhalten an, wenn sie unter Ber\"{u}cksichtigung aller
verf\"{u}gbaren Informationen (also dem privaten Signal und dem beobachteten
Verhalten ihrer Vorg\"{a}nger) meint, dass die Wahrscheinlichkeit von $B$
mehr als 50\% betr\"{a}gt. Wenn Person $i$ meint, dass die
Wahrscheinlichkeit von $B$ kleiner ist als 50\%, nimmt sie das Verhalten
nicht an. Bei genau 50\% wirft sie eine M\"{u}nze.
\begin{enumerate}
\item Skizzieren Sie den Modellaufbau in Form eines Wahrscheinlichkeitsbaums.
\item Welche bedingte Wahrscheinlichkeit \"{u}ber das Eintreten von $B$ hat
Person 1, wenn $S_{1}$ eintritt -- also nach dem Empfang eines positiven
Signals?
\item Wie wahrscheinlich ist es, dass Person 1 das Verhalten annimmt ($A_{1}$), 
obwohl es sch\"{a}dlich ist?
\item Welche bedingte Wahrscheinlichkeit \"{u}ber das Eintreten von $B$ hat
Person 2, wenn $S_{2}$ und $A_{1}$ eintreten?
\item Wie wahrscheinlich ist es, dass Person 2 das Verhalten annimmt ($A_{2}$%
), obwohl es sch\"{a}dlich ist, wenn zuvor schon Person 1 das Verhalten
angenommen hat?
\item Analysieren Sie nun die Personen 3 und 4.
\end{enumerate}

\item Penalty kicks and mixed strategies

This exercise is based on the article \textquotedblleft Professionals Play
Minimax\textquotedblright\ by Ignacio Palacios-Huerta, \emph{Review of
Economic Studies} 70 (2003) 395--415; this article is also nicely discussed
in the book \textquotedblleft Why England Lose\textquotedblright\ by Simon
Kuper and Stefan Szymanski, 2009. The article can be downloaded (password
protected pdf) from the internet site of this course.

A soccer penalty kick is a simple game theoretic situation for the kicker
(K) and the goalkeeper (G). We assume that the kicker has to decide whether
to kick to the right (R) or to the left (L), and that there are no
other possibilities than R and L. Similarly, the goalkeeper has to decide
whether to delve to the right or left. Players have a natural tendency to
kick to one side (usually right-footed players kick to the right and vice
versa). We call the kicker's stronger, preferred side the natural side and
the other one the unnatural side. To keep the notation more intuitive, and
without loss of generality, we assume that the natural side is right. The
scoring probabilities (calculated from a large database, but here we assume
these probabilities are known constants) depend on the decisions, e.g. if
the kicker kicks to his strong side (R) and the goalkeeper delves to the
other side (L), then the scoring probability is about 93\%.\footnote{The 
probabilities in the off-diagonal seem counter-intuitive, and have been
transposed in the book by Kuper and Szymanski. But the results (even in the
book) are in line with the table given here.}

\begin{center}
\begin{tabular}{cc|cc}
&  & \multicolumn{2}{|c}{Goalkeeper} \\
&  & L & R \\ \cline{2-4}
Kicker & L & 0.5830 & 0.9497 \\
& R & 0.9291 & 0.6992
\end{tabular}
\end{center}

The best strategy for the kicker is to randomize his kick, i.e. to kick to
the left with a certain probability $k_{L}$ and to the right with 
$k_{R}=1-k_{L}$. In the same way, the best strategy for the goalkeeper is to
randomize as well (with probabilities $g_{L}$ and $g_{R}$). In game theory,
these randomized strategies are called mixed strategies.

\begin{enumerate}
\item Determine the optimal mixed strategies for the kicker and the
goalkeeper.

Hints: In equilibrium, the goalkeeper chooses $g_{L}$ and $g_{R}$ such that
-- given the probabilities shown in the table above -- the scoring
probability for the kicker is the same, no matter if he chooses L or R.
Similarly, the kicker chooses $k_{L}$ and $k_{R}$ such that the goalkeeper
is indifferent between L and R (but remember that success is defined
the other way round for the goalkeeper).

\item Looking at the decisions actually taken by professional soccer
players, one finds that goalkeepers delve to the right with probability
57.69\% while kickers kick to the right with probability 60.02\%. Test if
their behavior is compatible with the theoretically optimal mixed strategy.
The number of penalty kicks analyzed for the paper (the sample size) was 
754 for goalkeepers and 808 for kickers. You may impose restricting 
assumptions (e.g.~as to the sampling mechanism).
\end{enumerate}
\end{enumerate}
