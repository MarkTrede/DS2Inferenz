
\documentclass{article}
%%%%%%%%%%%%%%%%%%%%%%%%%%%%%%%%%%%%%%%%%%%%%%%%%%%%%%%%%%%%%%%%%%%%%%%%%%%%%%%%%%%%%%%%%%%%%%%%%%%%%%%%%%%%%%%%%%%%%%%%%%%%%%%%%%%%%%%%%%%%%%%%%%%%%%%%%%%%%%%%%%%%%%%%%%%%%%%%%%%%%%%%%%%%%%%%%%%%%%%%%%%%%%%%%%%%%%%%%%%%%%%%%%%%%%%%%%%%%%%%%%%%%%%%%%%%
\usepackage{graphicx}
\usepackage{amsmath}
\usepackage[a4paper]{geometry}
\usepackage[german]{babel}

\setcounter{MaxMatrixCols}{10}
%TCIDATA{OutputFilter=LATEX.DLL}
%TCIDATA{Version=5.00.0.2570}
%TCIDATA{<META NAME="SaveForMode" CONTENT="1">}
%TCIDATA{Created=Wed Mar 07 14:49:20 2001}
%TCIDATA{LastRevised=Wednesday, January 16, 2008 12:28:33}
%TCIDATA{<META NAME="GraphicsSave" CONTENT="32">}
%TCIDATA{<META NAME="DocumentShell" CONTENT="General\Blank Document">}
%TCIDATA{CSTFile=LaTeX article (bright).cst}

\input{tcilatex}

\begin{document}

\title{\"{U}bungsblatt 10\thanks{%
Die mit einem Sternchen gekennzeichneten Aufgaben werden voraussichtlich im
Proseminar besprochen.}}
\author{Testverfahren}
\date{Wintersemester 2007/2008}
\maketitle

\begin{enumerate}
\item[$1.^{\ast }$] Die Zufallsvariablen $X_{1},\ldots ,X_{n}$ seien unabh%
\"{a}ngig und identisch normalverteilt mit Standardabweichung $\sigma =5$
und unbekanntem Erwartungswert $\mu $. Eine einfache Zufallsstichprobe vom
Umfang $n=81$ ergibt einen Stichprobenmittelwert von $\bar{x}=37$.

\begin{enumerate}
\item \"{U}berpr\"{u}fen Sie mittels eines geeigneten Testverfahrens die
Nullhypothese, dass der Erwartungswert gr\"{o}\ss er oder gleich 38 ist,
gegen die Alternative, dass der Erwartungswert kleiner als 38 ist
(Signifikanzniveau $\alpha =0.05$).

\item Nehmen Sie an, dass der Erwartungswert tats\"{a}chlich $\mu =37$ ist.
Wie gro\ss\ ist die Power des Tests? (Die Power ist die Wahrscheinlichkeit,
dass die falsche Nullhypothese als falsch erkannt wird.)
\end{enumerate}

\item[$2.^{\ast }$] Auf einer Maschine werden Werkst\"{u}cke hergestellt,
deren L\"{a}nge normalverteilt ist. Eine Zufallsstichprobe von 10 St\"{u}ck
ergab folgende Werte in mm (vgl. Aufgabe 3 von \"{U}bungsblatt 9):

%TCIMACRO{\TeXButton{BC}{\begin{center}}}%
%BeginExpansion
\begin{center}%
%EndExpansion
\begin{tabular}{|c|c|c|c|c|c|c|c|c|c|}
\hline
42.8 & 36.9 & 41.2 & 39.3 & 40.4 & 35.7 & 37.6 & 43.5 & 35.6 & 36.8 \\ \hline
\end{tabular}
%TCIMACRO{\TeXButton{EC}{\end{center}}}%
%BeginExpansion
\end{center}%
%EndExpansion

Testen Sie auf einem Signifikanzniveau von $\alpha =0.01$ die Nullhypothese,
dass der Erwartungswert mindestens 41~mm betr\"{a}gt.

\item[$3.^{\ast }$] Das Sozio-\"{o}konomische Panel enth\"{a}lt f\"{u}r eine
Stichprobe von Personen, die in Deutschland leben, Daten aus vielen
Bereichen des Lebens. Die Personen werden j\"{a}hrlich befragt, unter
anderem zum Arbeitseinkommen und zur Arbeitszeit. Wir betrachten im
Folgenden nur die M\"{a}nner, die vollzeitbesch\"{a}ftigt waren, einen
Hochschulabschluss (oder Fachhochschulabschluss) hatten und zwischen 30 und
35 Jahre alt waren. Die folgende Tabelle zeigt den durchschnittlichen
Stundenlohn sowie die korrigierte Standardabweichung des Stundenlohns f\"{u}%
r alleinstehende und verheiratete M\"{a}nner (Daten von 1997):

%TCIMACRO{\TeXButton{BC}{\begin{center}}}%
%BeginExpansion
\begin{center}%
%EndExpansion
\begin{tabular}{|l|r|r|}
\hline
& Verheiratete & Singles \\ \hline
Anzahl Beobachtungen & 89 & 57 \\ 
Durchschnittslohn & 33.59 & 30.05 \\ 
Standardabweichung & 13.50 & 12.34 \\ \hline
\end{tabular}
%TCIMACRO{\TeXButton{EC}{\end{center}}}%
%BeginExpansion
\end{center}%
%EndExpansion

\"{U}berpr\"{u}fen Sie mit einem statistischen Test, ob der Unterschied der
durchschnittlichen Stundenl\"{o}hne noch als zuf\"{a}llig aufgefasst werden
kann ($\alpha =0.05$).

\item[$4.^{\ast }$] Laden Sie den Datensatz \texttt{mrendite} und betrachten
Sie die Monatsrenditen der Volkswagenaktie. Sie wollen untersuchen, ob die
erwartete Rendite zeitlich konstant ist. Dazu unterteilen Sie den Zeitraum
in zwei Perioden: Oktober 1997 bis M\"{a}rz 2000 und April 2000 bis
September 2002. Die Renditen von Oktober 1997 bis M\"{a}rz 2000 ($n=30$
Beobachtungen) seien die Realisationen einer einfachen Stichprobe $%
X_{1},\dots ,X_{30}$; die Renditen von April 2000 bis September 2002 ($m=30$
Beobachtungen) seien die Realisationen einer einfachen Stichprobe $%
Y_{1},\ldots ,Y_{30}$.

Testen Sie auf einem Signifikanzniveau von $\alpha =0.05$ die Nullhypothese,
dass die erwartete Rendite $\mu _{X}$ der ersten Periode gleich der
erwarteten Renditen $\mu _{Y}$ der zweiten Periode war. Setzen Sie dabei
voraus, dass die Renditen normalverteilt sind und dass sich die
Standardabweichung nicht ge\"{a}ndert hat.

\item[$5.^{\ast }$] Folgende Meldung war am 26. April 1997 in der S\"{u}%
ddeutschen Zeitung:

\textbf{Schuh ahoi}

An den holl\"{a}ndischen Str\"{a}nden werden mehr linke als rechte Schuhe
angesp\"{u}lt. In Schottland ist es genau umgekehrt. Das hat eine
Untersuchung niederl\"{a}ndischer Biologen ergeben. Wie das Institut f\"{u}r
Wald- und Naturforschung in Wageningen mitteilte, fanden die Wissenschaftler
auf der holl\"{a}ndischen Nordseeinsel Texel 68 linke und 39 rechte Schuhe.
Auf den schottischen Shetlandinseln dagegen sammelten sie 63 linke und 93
rechte Schuhe ein. Mit der Untersuchung wollte der auf Meeresv\"{o}gel
spezialisierte Biologe Mardik Leopold beweisen, da\ss\ zwei Gegenst\"{a}nde
mit einer unterschiedlichen Form im Meer in verschiedene Richtungen treiben.
Deswegen sp\"{u}len nach seinen Erkenntnissen an bestimmten Str\"{a}nden
mehr rechte und an anderen mehr linke Muschelh\"{a}lften von Schalentieren
an. (dpa)

\begin{enumerate}
\item Testen Sie, ob der Anteil der auf Texel angesp\"{u}lten linken Schuhe
signifikant h\"{o}her als 50\% ist ($\alpha =0.05$).

\item Testen Sie, ob der Anteil der auf den Shetlandinseln angesp\"{u}lten
rechten Schuhe signifikant h\"{o}her als 50\% ist ($\alpha =0.05$).
\end{enumerate}

\item Eine gro\ss e Bank \"{u}berpr\"{u}ft den Erfolg ihrer
Direktwerbeaktion f\"{u}r einen Investmentfond: Sie schickt einen Werbebrief
an eine Testgruppe von 150000 zuf\"{a}llig ausgew\"{a}hlten Kunden und z\"{a}%
hlt, wie viele Kunden den Investmentfond kaufen. Au\ss erdem z\"{a}hlt sie,
wie viele von 150000 Kunden aus einer Kontrollgruppe, die den Werbebrief
nicht erhalten haben, den Investmentfond kaufen.

Von den 150000 Kunden der Testgruppe haben 2905 den Fond gekauft. Von den
150000 Kunden der Kontrollgruppe haben 2224 den Fond gekauft.

Testen Sie, ob die Abschlussquote der Testgruppe signifikant h\"{o}her ist
als die Abschlussquote der Kontrollgruppe ($\alpha =0.05$).

\item[$7.^{\ast }$] Diese Aufgabe basiert auf dem Fachartikel \quotedblbase
Gift Exchange in the Field\textquotedblright\ von Armin Falk, \emph{%
Econometrica}, vol. 75, pp. 1501-1511, 2007.\footnote{%
Einen Link auf den Artikel finden Sie auf der Internetseite der Vorlesung.}
In der Zusammenfassung hei\ss t es:

This study reports evidence from a field experiment that was conducted to
investigate the relevance of gift exchange in a natural setting. In
collaboration with a charitable organization, we sent roughly 10,000
solicitation letters to potential donors. One-third of the letters contained
no gift, one-third contained a small gift, and one-third contained a large
gift. Treatment assignment was random. The results confirm the economic
importance of gift exchange. Compared to the no gift condition, the relative
frequency of donations increased by 17 percent if a small gift was included
and by 75 percent for a large gift. The study extends the current body of
research on gift exchange, which is almost exclusively confined to
laboratory studies.

Tabelle I des Artikels enth\"{a}lt folgende Angaben:

%TCIMACRO{\TeXButton{BC}{\begin{center}}}%
%BeginExpansion
\begin{center}%
%EndExpansion
\begin{tabular}{lccc}
\hline\hline
& No Gift & Small Gift & Large Gift \\ \hline
Number of solicitation letters & 3262 & 3237 & 3347 \\ 
Number of donations & 397 & 465 & 691 \\ 
Relative frequency of donations & 0.12 & 0.14 & 0.21 \\ \hline
\end{tabular}%
%TCIMACRO{\TeXButton{EC}{\end{center}}}%
%BeginExpansion
\end{center}%
%EndExpansion

\begin{enumerate}
\item Testen Sie, ob die Wahrscheinlichkeit einer Spende bei einem kleinen
Geschenk signifikant h\"{o}her ist als die Wahrscheinlichkeit einer Spende
ohne Geschenk.

\item Testen Sie, ob die Wahrscheinlichkeit einer Spende bei einem gro\ss en
Geschenk signifikant h\"{o}her ist als die Wahrscheinlichkeit einer Spende
bei einem kleinen Geschenk.
\end{enumerate}

\item Die Daten des Sozio-\"{o}konomischen Panels enthalten Angaben zur
allgemeinen Lebens\-zufriedenheit. Die Zufriedenheit wird auf einer Skala
von 0 (sehr unzufrieden) bis 10 (sehr zufrieden) angegeben. Sie ist im
Folgenden umkodiert zu niedrig (Werte 0--3), mittel (Werte 4--6), hoch
(Werte 7--8) und sehr hoch (Werte 9--10). Die folgende Tabelle gibt f\"{u}r M%
\"{a}nner und Frauen die Zufriedenheit an (Daten f\"{u}r 1998).

%TCIMACRO{\TeXButton{BC}{\begin{center}}}%
%BeginExpansion
\begin{center}%
%EndExpansion
\begin{tabular}{|c|rrrr|}
\hline
& \multicolumn{4}{|c|}{Zufriedenheit} \\ \cline{2-5}
Geschlecht & niedrig & mittel & hoch & sehr hoch \\ \hline
M\"{a}nner & 344 & 1925 & 3802 & 1012 \\ 
Frauen & 329 & 2189 & 3892 & 1160 \\ \hline
\end{tabular}%
%TCIMACRO{\TeXButton{EC}{\end{center}}}%
%BeginExpansion
\end{center}%
%EndExpansion

Testen Sie, ob Zufriedenheit und Geschlecht stochastisch unabh\"{a}ngig sind
($\alpha =0.05$).

\item[$9.^{\ast }$] Laden Sie den Datensatz \texttt{mrendite} (im Text- oder
Excelformat) und betrachten Sie die Renditen von BASF und Bayer. F\"{u}llen
Sie die folgende Tabelle aus:

%TCIMACRO{\TeXButton{BC}{\begin{center}}}%
%BeginExpansion
\begin{center}%
%EndExpansion
\begin{tabular}{|c||c|c|}
\hline
& \multicolumn{2}{|c|}{Bayer} \\ \hline
BASF & positive Rendite & negative Rendite \\ \hline\hline
positive Rendite &  &  \\ \hline
negative Rendite &  &  \\ \hline
\end{tabular}
%TCIMACRO{\TeXButton{EC}{\end{center}}}%
%BeginExpansion
\end{center}%
%EndExpansion

Testen Sie mit Hilfe der Tabelleneintr\"{a}ge, ob die Renditen stochastisch
unabh\"{a}ngig sind ($\alpha =0.05$).

\item In der ZEIT\ vom 20.12.2000 erschienen die Ergebnisse einer Studie zum
Rechtsextremismus in Deutschland. Die folgende Tabelle gibt einige Zahlen
wieder: Die Gesamtzahl der Befragten ist nicht gegeben; setzen Sie f\"{u}r
die alten Bundesl\"{a}nder $n_{alt}=1200$ und f\"{u}r die neuen Bundesl\"{a}%
nder $n_{neu}=900$. Angegeben sind die Anteil der Antworten mit
\textquotedblleft stimme voll und ganz zu\textquotedblright\ oder
\textquotedblleft stimme eher zu\textquotedblright .

%TCIMACRO{\TeXButton{BC}{\begin{center}}}%
%BeginExpansion
\begin{center}%
%EndExpansion
\begin{tabular}{|l|c|c|}
\hline
& \multicolumn{2}{|c|}{Stimme zu (in \%)} \\ 
& Alte Bundesl. & Neue Bundesl. \\ \hline
Nur wer etwas leistet, soll auch etwas verdienen & 73 & 73 \\ \hline
Wem es bei uns schlecht geht, der ist selbst schuld & 43 & 32 \\ \hline
Ich finde Homosexuelle absto\ss end und pervers & 37 & 32 \\ \hline
Abtreibungen sollten grunds\"{a}tzlich verboten werden & 27 & 15 \\ \hline
Die Todesstrafe sollte wieder eingef\"{u}hrt werden & 20 & 33 \\ \hline
Behinderte sind eine Belastung f\"{u}r die Gesellschaft & 14 & 6 \\ \hline
\end{tabular}%
%TCIMACRO{\TeXButton{EC}{\end{center}}}%
%BeginExpansion
\end{center}%
%EndExpansion

Testen Sie auf einem Signifikanzniveau von $\alpha =0.05$, ob das
Gesellschaftsbild in West- und Ostdeutschland signifikant unterschiedlich
ist.

\item[$11.^{\ast }$] Der Physiker Frank Benford stellte 1920 fest, dass die
vorderen Seiten seines Logarithmen-Buches st\"{a}rker abgegriffen waren als
die hinteren Seiten. Die ersten Seiten gaben die Logarithmen der Zahlen mit
niedrigen ersten Ziffern wieder (beispielsweise ist die erste Ziffer der
Zahl 246981 die 2). Benford stellte die Hypothese auf, dass er die Zahlen
mit niedrigen ersten Ziffern h\"{a}ufiger nachschlug, weil es in der Welt
mehr Zahlen mit niedriger Anfangs\-ziffer gibt, als solche mit einer hohen
Anfangsziffer.

Die Zufallsvariable $X$ sei die erste Ziffer einer zuf\"{a}llig ausgew\"{a}%
hlten Zahl (ohne f\"{u}hrende Nullen); es kann sich beispielsweise um die L%
\"{a}nge eines Flusses in Meilen, die Anzahl der Einwohner einer Hauptstadt
oder das Sozialprodukt eines Landes in Euro handeln etc. Das Benfordsche
Gesetz postuliert 
\begin{equation*}
P\left( X=d\right) =\log _{10}\left( 1+\frac{1}{d}\right)
\end{equation*}%
f\"{u}r $d=1,\ldots ,9$. Die folgende Tabelle zeigt in der zweiten Spalte
die Wahrscheinlichkeiten $P\left( X=d\right) $.

%TCIMACRO{\TeXButton{BC}{\begin{center}}}%
%BeginExpansion
\begin{center}%
%EndExpansion
\begin{tabular}{|c|c|c|}
\hline
Ziffer $d$ & $P\left( X=d\right) $ & $n_{d}$ \\ \hline\hline
1 & 0.3010 & 26 \\ \hline
2 & 0.1761 & 20 \\ \hline
3 & 0.1249 & 14 \\ \hline
4 & 0.0969 & 8 \\ \hline
5 & 0.0792 & 6 \\ \hline
6 & 0.0669 & 15 \\ \hline
7 & 0.0580 & 2 \\ \hline
8 & 0.0512 & 6 \\ \hline
9 & 0.0458 & 3 \\ \hline
\end{tabular}
%TCIMACRO{\TeXButton{EC}{\end{center}}}%
%BeginExpansion
\end{center}%
%EndExpansion

Jemand schreibt die ersten 100 Zahlen, die in einer Zeitung vorkommen, auf
und untersucht ihre Anfangsziffern. Die dritte Spalte der Tabelle zeigt die
absoluten H\"{a}ufigkeiten.

Testen Sie auf einem Signifikanzniveau von $\alpha =0.05$ die Nullhypothese,
dass diese 100 Zahlen mit dem Benfordschen Gesetz vereinbar sind.

\item Laden Sie den Datensatz \texttt{mrendite.csv}. Testen Sie, ob die
DAX100-Rendite durch eine Normalverteilung mit Erwartungswert $\mu =-0.2$
und Standardabweichung $\sigma =7.5$ beschrieben werden kann ($\alpha =0.05$%
). Benutzen Sie die 0.2-, 0.4-, 0.6- und 0.8-Quantile der $N\left(
-0.2,7.5^{2}\right) $ f\"{u}r die Partitionierung.
\end{enumerate}

\end{document}
