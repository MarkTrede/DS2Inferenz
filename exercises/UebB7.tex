\section[Gemeinsame Verteilungen]{Gemeinsame Verteilungen}

\begin{enumerate}
\item Risikoverringerung durch Diversifikation I:

Ein Anleger legt 50\% seines Verm\"{o}gens in Aktien A an und 50\% in Aktien
B. Die Renditen $R_{A}$ und $R_{B}$ der beiden Aktien f\"{u}r das folgende
Jahr sind Zufallsvariablen. Ihre Erwartungswerte sind 
\begin{eqnarray*}
E\left( R_{A}\right) &=&0.15 \\
E\left( R_{B}\right) &=&0.13.
\end{eqnarray*}%
Die Volatilit\"{a}t der Rendite wird meist durch die Standardabweichung
(oder die Varianz) gemessen; die Standardabweichungen sind 
\begin{eqnarray*}
\sqrt{Var\left( R_{A}\right) } &=&0.20 \\
\sqrt{Var\left( R_{B}\right) } &=&0.17.
\end{eqnarray*}%
Die beiden Aktien bewegen sich nicht unabh\"{a}ngig voneinander; ihre
Kovarianz betr\"{a}gt 
\begin{equation*}
Cov\left( R_{A},R_{B}\right) =0.024.
\end{equation*}
\begin{enumerate}
\item Berechnen Sie den Korrelationskoeffizienten f\"{u}r $R_{A}$ und $R_{B}$.
\item Berechnen Sie Erwartungswert und Standardabweichung der Rendite des
Portfolios. Hinweis: Die Portfoliorendite betr\"{a}gt $R_{P}=0.5\cdot
R_{A}+0.5\cdot R_{B}$.
\item Kann der Anleger durch eine andere Verm\"{o}gensaufteilung sein Risiko
verringern? Welche Aufteilung minimiert die Varianz der Portfoliorendite?
\end{enumerate}

\item Risikoverringerung durch Diversifikation II:

Betrachten Sie wieder die Jahresrenditen zweier Aktien $A$ und $B$. Die
Zufallsvariable $R_{A}$ beschreibt die Verteilung der Jahresrendite der
Aktie $A$, analog $R_{B}$. Die erwarteten Renditen der beiden Aktien sind%
\begin{equation*}
\begin{array}{ll}
E(R_{A})=\mu _{A} & \quad E(R_{B})=\mu _{B} \\ 
Var(R_{A})=\sigma _{A}^{2} & \quad Var(R_{B})=\sigma _{B}^{2}.%
\end{array}%
\end{equation*}%
Der Korrelationskoeffizient betr\"{a}gt $\rho $. Eine Anlegerin teilt ihr
gesamtes Verm\"{o}gen auf diese beiden Aktien auf. Der Anteil ihres Verm\"{o}%
gens, der auf die Aktie $A$ entf\"{a}llt, ist $w$. Entsprechend wird ein
Anteil von $1-w$ des Verm\"{o}gens in die Aktie $B$ investiert. Gehen Sie
davon aus, dass $0\leq w\leq 1$ ist. Die Portfoliorendite $R_{P}$ ist%
\begin{equation*}
R_{P}=wR_{A}+(1-w)R_{B}.
\end{equation*}
Zeigen Sie, dass die gewichtete durchschnittliche Standardabweichung der
Einzelrenditen nie kleiner als die Standardabweichung der Portfoliorendite
ist.

\item Nichtlineare Zusammenh\"{a}nge:

Die Kovarianz und auch der Korrelationskoeffizient messen nur die St\"{a}rke
des linearen Zusammenhangs zweier Zufallsvariablen. Es kann durchaus
passieren, dass es zwar keinen linearen, aber dennoch einen starken --
eventuell sogar perfekten -- nichtlinearen Zusammenhang gibt. Sei $X\sim
N\left( 0,1\right) $. Definiere die Zufallsvariable $Y=X^{2}$. Zeigen Sie,
dass 
\begin{equation*}
Cov\left( X,Y\right) =0
\end{equation*}%
ist. Hinweis: F\"{u}r $X\sim N\left( 0,1\right) $ gilt wegen der Symmetrie
um den Nullpunkt $E\left( X^{3}\right) =0.$

\item Sei $Y=X_{1}+\ldots +X_{K}$. Zeigen Sie, dass $Var\left(
Y\right) =\sum_{i=1}^{K}Cov\left( X_{i},Y\right) .$

\item Zeigen Sie, dass die folgenden Gleichungen gelten (die
erste Gleichung ist nat\"{u}rlich eine Definitionsgleichung):%
\begin{eqnarray*}
Cov(X,Y) &=&E\left[ \left( X-E(X)\right) \left( Y-E(Y)\right) \right]  \\
&=&E(XY)-E(X)E(Y) \\
&=&E\left[ \left( X-E(X)\right) \left( Y-a\right) \right]  \\
&=&E\left[ \left( X-a\right) \left( Y-E(Y)\right) \right]  \\
&=&E\left[ \left( X-E(X)\right) Y\right]  \\
&=&E\left[ X\left( Y-E(Y)\right) \right] 
\end{eqnarray*}%
f\"{u}r $a\in \mathbb{R}.$

\item Seien $X$ und $Y$ gemeinsam stetig verteilte Zufallsvariablen mit
der gemeinsamen Verteilungsfunktion%
\begin{equation*}
F_{XY}(x,y)=\left( x^{-2}+y^{-2}-1\right) ^{-1/2}
\end{equation*}%
f\"{u}r $0<x,y\leq 1$. F\"{u}r Werte von $x$ und $y$ au\ss erhalb dieses
Bereichs geht man wie folgt vor: Falls $x>1$ setzt man $x=1$ und analog f%
\"{u}r $y$. F\"{u}r $x,y\leq 0$ ist $F_{XY}(x,y)=0.$
\begin{enumerate}
\item Bestimmen Sie die Randverteilungsfunktion von $X$.
\item Sind $X$ und $Y$ stochastisch unabh\"{a}ngig?
\item Bestimmen Sie die Randdichten von $X$ und $Y$.
\item Bestimmen Sie die gemeinsame Dichtefunktion von $X$ und $Y$.
\item Bestimmen Sie die bedingte Dichtefunktion von $X$ unter der Bedingung $%
Y=y$. Zeich\-nen Sie $f_{X|Y=0.1}(x)$ und $f_{X|Y=0.8}(x)$ im Intervall $%
[0,1]$.
\end{enumerate}

\item Ein Ein-Faktor-Modell f\"{u}r den Aktienmarkt:

Ein Portfolio bestehe aus $K$ Wertpapieren. Die Rendite $R_{i}$ des
Wertpapiers $i$ f\"{u}r das kommende Jahr ist eine Zufallsvariable, $%
i=1,\ldots ,K$. Betrachten Sie das folgende einfache multivariate
Renditemodell. Die Rendite $R_{i}$ setzt sich additiv aus einer
Marktkomponente $Y$ (oft auch Faktor oder Marktfaktor genannt) und einer
individuellen Komponente $X_{i}$ zusammen, also%
\begin{equation*}
R_{i}=Y+X_{i}
\end{equation*}%
mit $Y\sim N\left( \mu _{Y},\sigma _{Y}^{2}\right) $ und $X_{i}\sim N\left(
0,\sigma _{X}^{2}\right) $ f\"{u}r $i=1,\ldots ,K$. Die individuellen
Komponenten $X_{1},\ldots ,X_{K}$ sind (global) unabh\"{a}ngig und auch unabh%
\"{a}ngig von $Y$, sie haben alle die gleiche Varianz $\sigma _{X}^{2}.$
\begin{enumerate}
\item Berechnen Sie die Varianz von $R_{i}$ (f\"{u}r ein beliebiges $i$).
\item Berechnen Sie Kovarianz von $R_{i}$ und $R_{j}$ f\"{u}r $i\neq j$.
\item Berechnen Sie den Korrelationskoeffizienten von $R_{i}$ und $R_{j}$ f%
\"{u}r $i\neq j$.
\end{enumerate}

\item Die Geld-Brief-Spanne oder: Bid-Ask-Spread

Auf manchen Aktienm\"{a}rkten wird Liquidit\"{a}t durch Market-Maker
(manchmal auch: Sponsoren, Betreuer) bereitgestellt. Market-Maker sind
verpflichtet, Kauf- und Verkaufsgebote (bid and ask orders) zu erf\"{u}llen.
Sie kaufen jedoch zu einem niedrigeren Kurs als sie verkaufen. Die Differenz
aus dem Geld- und dem Briefkurs nennt man Geld-Brief-Spanne
(Bis-Ask-Spread). Der \quotedblbase wahre\textquotedblright\ Wert der Aktie
ist die Mitte zwischen dem Geld- und dem Briefkurs. In einem wichtigen
Artikel\footnote{%
Richard Roll (1984), A Simple Implicit Measure of the Effective Bid-Ask
Spread in an Efficient Market, Journal of Finance 39: 1127-1139. Sie finden
einen Link auf den passwortgesch\"{u}tzten Artikel auf der Internet-Seite
der Vorlesung.} hat Richard Roll gezeigt, dass durch einen Market-Maker eine
negative Korrelation zwischen aufeinanderfolgenden Kursver\"{a}nderungen
induziert wird, selbst wenn keinerlei neue Informationen eintreffen. Zur
Vereinfachung geht Roll davon aus, dass Geld- und Brieforders gleich
wahrscheinlich sind und dass aufeinanderfolgende Ordertypen unabh\"{a}ngig
voneinander sind.

Gegeben, dass die letzte Transaktion im Zeitpunkt $t-1$ eine Geldorder war,
sind folgende weiteren Kursverl\"{a}ufe f\"{u}r die Zeitpunkte $t$ und $t+1$
m\"{o}glich (die Abbildung ist dem Artikel von Roll entnommen):

%TCIMACRO{\TeXButton{BC}{\begin{center}}}%
%BeginExpansion
\begin{center}%
%EndExpansion
\unitlength1cm 
\begin{picture}(9,4)
\put(0,1){Bid Price}
\put(0,3){Ask Price}
\put(2,1.1){\circle*{0.1}}
\put(2,1.1){\vector(1,0){3}}
\put(2,1.1){\vector(3,2){3}}
\put(5,1.1){\circle*{0.1}}
\put(5,3.1){\circle*{0.1}}
\put(5,1.1){\vector(3,2){3}}
\put(5,1.1){\vector(1,0){3}}
\put(5,3.1){\vector(1,0){3}}
\put(5,3.1){\vector(3,-2){3}}
\put(8,1.1){\circle*{0.1}}
\put(8,3.1){\circle*{0.1}}
\multiput(2,2.1)(0.2,0){35}{\line(1,0){0.1}}
\multiput(2,3.1)(0.2,0){15}{\line(1,0){0.1}}
\put(9,2){Value}
\put(1.8,0.6){$t-1$}
\put(5,0.6){$t$}
\put(7.8,0.6){$t+1$}
\end{picture}%
%TCIMACRO{\TeXButton{EC}{\end{center}}}%
%BeginExpansion
\end{center}%
%EndExpansion

\begin{enumerate}
\item Erg\"{a}nzen Sie in der folgenden Tabelle die bedingte gemeinsame
Wahr\-schein\-lich\-keits\-ver\-tei\-lung f\"{u}r $\Delta p_{t}=p_{t}-p_{t-1}
$ und $\Delta p_{t+1}=p_{t+1}-p_{t}$ unter der Bedingung, dass die letzte
Transaktion eine Geldorder war.

%TCIMACRO{\TeXButton{BC}{\begin{center}}}%
%BeginExpansion
\begin{center}%
%EndExpansion
\begin{tabular}{c|ccc}
& \multicolumn{3}{|c}{$\Delta p_{t+1}$} \\ \hline
$\Delta p_{t}$ & $-s$ & $0$ & $+s$ \\ \hline
$-s$ & \rule[-0.15cm]{0cm}{0.6cm}\qquad & \qquad & \qquad \\ 
$0$ & \rule[-0.15cm]{0cm}{0.6cm} &  &  \\ 
$+s$ & \rule[-0.15cm]{0cm}{0.6cm} &  & 
\end{tabular}%
%TCIMACRO{\TeXButton{EC}{\end{center}}}%
%BeginExpansion
\end{center}%
%EndExpansion

\item Erg\"{a}nzen Sie in der folgenden Tabelle die bedingte gemeinsame
Wahr\-schein\-lich\-keits\-ver\-tei\-lung f\"{u}r $\Delta p_{t}=p_{t}-p_{t-1}
$ und $\Delta p_{t+1}=p_{t+1}-p_{t}$ unter der Bedingung, dass die letzte
Transaktion eine Brieforder war.

%TCIMACRO{\TeXButton{BC}{\begin{center}}}%
%BeginExpansion
\begin{center}%
%EndExpansion
\begin{tabular}{c|ccc}
& \multicolumn{3}{|c}{$\Delta p_{t+1}$} \\ \hline
$\Delta p_{t}$ & $-s$ & $0$ & $+s$ \\ \hline
$-s$ & \rule[-0.15cm]{0cm}{0.6cm}\qquad  & \qquad  & \qquad  \\ 
$0$ & \rule[-0.15cm]{0cm}{0.6cm} &  &  \\ 
$+s$ & \rule[-0.15cm]{0cm}{0.6cm} &  & 
\end{tabular}%
%TCIMACRO{\TeXButton{EC}{\end{center}}}%
%BeginExpansion
\end{center}%
%EndExpansion

\item Erg\"{a}nzen Sie in der folgenden Tabelle die bedingte gemeinsame
Wahr\-schein\-lich\-keits\-ver\-tei\-lung f\"{u}r $\Delta p_{t}=p_{t}-p_{t-1}
$ und $\Delta p_{t+1}=p_{t+1}-p_{t}$.

%TCIMACRO{\TeXButton{BC}{\begin{center}}}%
%BeginExpansion
\begin{center}%
%EndExpansion
\begin{tabular}{c|ccc}
& \multicolumn{3}{|c}{$\Delta p_{t+1}$} \\ \hline
$\Delta p_{t}$ & $-s$ & $0$ & $+s$ \\ \hline
$-s$ & \rule[-0.15cm]{0cm}{0.6cm}\qquad  & \qquad  & \qquad  \\ 
$0$ & \rule[-0.15cm]{0cm}{0.6cm} &  &  \\ 
$+s$ & \rule[-0.15cm]{0cm}{0.6cm} &  & 
\end{tabular}%
%TCIMACRO{\TeXButton{EC}{\end{center}}}%
%BeginExpansion
\end{center}%
%EndExpansion

\item Berechnen Sie die Kovarianz von $\Delta p_{t}$ und $\Delta p_{t+1}$
als Funktion der Geld-Brief-Spanne.
\end{enumerate}

\item In einer Urne sind 20 wei\ss e und 30 schwarze Kugeln:
\begin{enumerate}
\item Jemand zieht mit Zur\"{u}cklegen 10 Kugeln aus der Urne. Wie gro\ss\ %
ist die Wahrscheinlichkeit, dass von den 10 gezogenen Kugeln mindestens 3,
aber h\"{o}chstens 5 wei\ss\ sind?
\item Jemand zieht mit Zur\"{u}cklegen 1000 Kugeln aus der Urne. Wie gro\ss\ %
ist die Wahrscheinlichkeit, dass von den 1000 gezogenen Kugeln mindestens
300, aber h\"{o}chstens 500 wei\ss\ sind?
\item Jemand zieht mit Zur\"{u}cklegen 1000 Kugeln aus der Urne. Wie gro\ss\ %
ist die Wahrscheinlichkeit, dass von den 1000 gezogenen Kugeln mindestens
380, aber h\"{o}chstens 420 wei\ss\ sind?
\end{enumerate}
\end{enumerate}
