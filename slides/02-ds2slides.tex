\documentclass[12pt,show notes]{beamer} 
%\documentclass[12pt,handout]{beamer}

\usepackage{beamerthemesplit}
\usepackage{amsmath}
\usepackage{amsfonts}
\usepackage{graphicx}
\usepackage{color}
\usepackage{eurosym}
\usepackage[ngerman]{babel}
\usepackage[ansinew]{inputenc}

\usetheme{Rochester}
\usecolortheme{beaver}
\setbeamertemplate{navigation symbols}{} 
\setbeamertemplate{footline}[text line]{} 
\graphicspath{{../plots/}}

\begin{document}

\title{Data Science 2}
\subtitle{Wahrscheinlichkeit}
\author{Prof.~Dr.~Mark Trede}
\institute[]{Institut f\"ur \"Okonometrie und Wirtschaftsstatistik}
\date{Oktober 2023}
\maketitle

\begin{frame}
\frametitle{Wahrscheinlichkeit}
\framesubtitle{Definition}
\begin{block}{Wahrscheinlichkeit}
Eine Abbildung
\[ P:A\longmapsto P(A) \]
hei�t Wahrscheinlichkeit, wenn gilt:
\begin{enumerate}
\item Nichtnegativit�t: F�r alle $A$ gilt $P(A) \geq 0$
\item Normierung: $P(\Omega)=1$
\item Additivit�t: F�r $A\cap B=\varnothing$ gilt 
\[ P(A\cup B)=P(A)+P(B) \]
\end{enumerate}
\end{block}
Die drei Bedingungen sind Axiome.
\end{frame}

\begin{frame}
\frametitle{Wahrscheinlichkeit}
\framesubtitle{Definition}
Aus den Axiome folgt z.B.\\\medskip
\begin{itemize}
\item Komplement�rereignis
\[ P(\bar{A})=1-P(A) \]
\item Additionssatz
\[ P(A\cup B)=P(A)+P(B)-P(A\cap B) \]
\end{itemize}
\end{frame}

\begin{frame}
\frametitle{Wahrscheinlichkeit}
\framesubtitle{Definition}
\begin{exampleblock}{Beispiel: Zwei Ereignisse} 
Gegeben seien zwei Ereignisse $A$ und $B$ mit den Wahrscheinlichkeiten
\begin{itemize}
\item $P(A)=0.6$
\item $P(B)=0.5$
\item $P(A\text{ oder } B)=P(A\cup B)=0.9$
\end{itemize}
\end{exampleblock}
\end{frame}

\begin{frame}
\frametitle{Wahrscheinlichkeit}
\framesubtitle{Definition}
\begin{exampleblock}{Forts. Beispiel: Zwei Ereignisse}
$P(A)=0.6, P(B)=0.5, P(A\cup B)=0.9$\\[1ex]
Die Wahrscheinlichkeit, dass $A$ und $B$ beide eintreten, ist
\uncover<2>{
\begin{align*}
P(A\cap B) &= P(A)+P(B)- P(A\cup B)\\
&= 0.6+0.5-0.9 \\
&= 0.2
\end{align*}
}
\end{exampleblock}
\end{frame}

\begin{frame}
\frametitle{Wahrscheinlichkeit}
\framesubtitle{Definition}
\begin{exampleblock}{Forts. Beispiel: Zwei Ereignisse}
$P(A)=0.6, P(B)=0.5, P(A\cup B)=0.9$\\[1ex]
Die Wahrscheinlichkeit, dass weder $A$ noch $B$ eintritt, ist
\uncover<2>{
\begin{align*}
P(\bar A\cap \bar B) &= P(\overline{A\cup B})\\
&= 1-P(A\cup B)\\
&= 1-0.9 \\
&= 0.1
\end{align*}
}
\end{exampleblock}
\end{frame}

\begin{frame}
\frametitle{Wahrscheinlichkeit}
\framesubtitle{Definition}
\begin{exampleblock}{Forts. Beispiel: Zwei Ereignisse}
$P(A)=0.6, P(B)=0.5, P(A\cup B)=0.9, P(A\cap B)=0.2$\\[1ex]
Die Wahrscheinlichkeit, dass entweder $A$ oder $B$ eintritt\\ (aber nicht beide), ist
\uncover<2>{
\begin{align*}
P((A\cup B)\backslash (A\cap B)) &= P(A\cup B)-P((A\cup B)\cap (A\cap B))\\
&= P(A\cup B)-P(A\cap B)\\
&= 0.9-0.2\\
&= 0.7
\end{align*}
}
\end{exampleblock}
\end{frame}

\begin{frame}
\frametitle{Wahrscheinlichkeit}
\framesubtitle{Laplace-Experimente}
\begin{block}{Laplace-Experimente}
Ein Zufallsexperiment hei�t Laplace-Experiment, wenn
$\Omega $ endlich ist und wenn alle Elementarereignisse
als gleich wahrscheinlich angenommen werden k�nnen.
\end{block}
\medskip
F�r jedes $\omega\in\Omega$ gilt $P(\{\omega\})=1/n$
\end{frame}

\begin{frame}
\frametitle{Wahrscheinlichkeit}
\framesubtitle{Laplace-Experimente}
\begin{itemize}
\item Laplace-Wahrscheinlichkeit: F�r $A\subset \Omega $ gilt 
\[P(A)=\frac{|A|}{|\Omega|}=\frac{\text{Anzahl der g�nstigen Ergebnisse}}{\text{Anzahl der m�glichen Ergebnisse}} \]
\item Zur Berechnung der Wahrscheinlichkeit muss 
man die g�nstigen Ergebnisse und die m�glichen Ergebnisse z�hlen
\end{itemize}
\end{frame}

\begin{frame}
\frametitle{Wahrscheinlichkeit}
\framesubtitle{Laplace-Experimente}
\begin{exampleblock}{Beispiel: W�rfelwurf}
Ergebnismenge $\Omega =\{1,2,3,4,5,6\}$\medskip
\begin{itemize}
\item Jedes Elementarereignis hat die Wahrscheinlichkeit $\frac{1}{6}$
\item Wahrscheinlichkeit von $A$: "`Eine gerade Zahl w�rfeln"'
\uncover<2>{
\[P(A)=\frac{|A|}{|\Omega|}=\frac{|\{2,4,6\}|}{|\{1,2,3,4,5,6\}|}=\frac{3}{6}=\frac{1}{2}\]
}
\end{itemize}
\end{exampleblock}
\end{frame}

\begin{frame}
\frametitle{Wahrscheinlichkeit}
\framesubtitle{Bedingte Wahrscheinlichkeit}
\begin{block}{Bedingte Wahrscheinlichkeit}
F�r $A,B$ mit $P(B)>0$ hei�t
\[ P(A|B)=\frac{P(A\cap B)}{P(B)} \]
die bedingte Wahrscheinlichkeit von $A$ unter der Bedingung $B$
\end{block}
\end{frame}

\begin{frame}
\frametitle{Wahrscheinlichkeit}
\framesubtitle{Bedingte Wahrscheinlichkeit}
\begin{itemize}
\item Wie ist $P(A)$ zu modifizieren, wenn man wei� oder annimmt, dass $B$ eingetreten ist?
\item Die bedingte Wahrscheinlichkeit $P(\cdot|B)$ ist eine Wahrscheinlichkeit im Sinne der Definition
\item Die �blichen Rechenregeln f�r Wahrscheinlichkeiten gelten weiterhin,
z.B.\ $P(A|B)=1-P(\bar A|B)$
\end{itemize}
\end{frame}

\begin{frame}
\frametitle{Wahrscheinlichkeit}
\framesubtitle{Bedingte Wahrscheinlichkeit}
\textbf{Multiplikationssatz f�r Wahrscheinlichkeiten:}\medskip
\begin{itemize}
\item Aus
\[ P(A|B)=\frac{P(A\cap B)}{P(B)} \]
folgt
\[ P(A\cap B)=P(A)P(B|A) \]
\item Drei Ereignisse
\[ P(A\cap B\cap C)=P(A)P(B|A)P(C|A\cap B) \]
\end{itemize}
\end{frame}

\begin{frame}
\frametitle{Wahrscheinlichkeit}
\framesubtitle{Bedingte Wahrscheinlichkeit}
\begin{exampleblock}{Beispiel: �berlebenswahrscheinlichkeit}
\begin{itemize}
\item Ereignisse: $A_i$ ``Person �berlebt Jahr $i$'' f�r $i=1,2,3$
\item Gegeben seien die (bedingten) Wahrscheinlichkeiten
\begin{align*}
P(A_1) &= 0.8 \\
P(A_2|A_1) &= 0.9\\
P(A_3|A_2)&= 0.95
\end{align*}
\item Wie hoch ist die Wahrscheinlichkeit, dass die Person (mindestens) bis zum Ende von Jahr 3 �berlebt?
\end{itemize}
\end{exampleblock}
\end{frame}

\begin{frame}
\frametitle{Wahrscheinlichkeit}
\framesubtitle{Bedingte Wahrscheinlichkeit}
\begin{exampleblock}{Forts. Beispiel: �berlebenswahrscheinlichkeit}
Die Wahrscheinlichkeit, dass die Person (mindestens) bis zum Ende von Jahr 3 �berlebt, betr�gt
\uncover<2>{
\begin{align*}
P(A_1\cap A_2\cap A_3) &= P(A_1)P(A_2|A_1)P(A_3|A_2\cap A_1) \\
&= 0.8 \times 0.9\times 0.95\\
&=0.684
\end{align*}
}
\end{exampleblock}
\end{frame}

\begin{frame}
\frametitle{Wahrscheinlichkeit}
\framesubtitle{Bedingte Wahrscheinlichkeit}
\begin{block}{Partition}
Die Ereignisse $A_1,\ldots ,A_n$ bilden eine Partition\\ 
(vollst�ndige Zerlegung) von $\Omega$, wenn 
\begin{itemize}
\item $\bigcup_{i=1}^{n}A_{i}=\Omega $ 
\item und $A_{i}\cap A_{j}=\varnothing$ f�r $i\neq j$ 
\item und $A_{i}\neq \varnothing $ f�r alle $i=1,\ldots ,n$
\end{itemize}
\end{block}
\end{frame}

\begin{frame}
\frametitle{Wahrscheinlichkeit}
\framesubtitle{Bedingte Wahrscheinlichkeit}
\begin{center}
\includegraphics[width=7cm]{../book/images/AdobeStock_101998556b_ButcherGuide.jpeg}
\end{center}
\end{frame}

\begin{frame}
\frametitle{Wahrscheinlichkeit}
\framesubtitle{Totale Wahrscheinlichkeit}
\textbf{Satz von der totalen Wahrscheinlichkeit}\medskip
\begin{itemize}
\item Sei\ $A_1,\ldots ,A_n$ eine Partition von $\Omega$
und $B$ ein Ereignis
\item Dann gilt
\[ P(B) = \sum_{i=1}^{n}P(B|A_i)\cdot P(A_i) \]
\item Gewichtete Summe der bedingten Wahrscheinlichkeiten
\end{itemize}
\end{frame}

\begin{frame}
\frametitle{Wahrscheinlichkeit}
\framesubtitle{Totale Wahrscheinlichkeit}
Begr�ndung: Aus $B=(B\cap A_1)\cup\ldots\cup (B\cap A_n)$ folgt
\begin{align*}
P(B) &= P\left(\bigcup_{i=1}^n (B\cap A_i)\right) \\
&= \sum_{i=1}^n P(B\cap A_i)\\
&= \sum_{i=1}^n P(B|A_i)P(A_i)
\end{align*}
\end{frame}

\begin{frame}
\frametitle{Wahrscheinlichkeit}
\framesubtitle{Totale Wahrscheinlichkeit}
\begin{exampleblock}{Beispiel: Totale Wahrscheinlichkeit}
Eine Person wird zuf�llig aus einer Population ausgew�hlt
\begin{align*}
B &\text{: Person ist arbeitslos } \\
A_{1} &\text{: Person hat keine/niedrige Ausbildung} \\
A_{2} &\text{: Person hat mittlere Ausbildung} \\
A_{3} &\text{: Person hat hohe Ausbildung}
\end{align*}
\end{exampleblock}
\end{frame}

\begin{frame}
\frametitle{Wahrscheinlichkeit}
\framesubtitle{Totale Wahrscheinlichkeit}
\begin{exampleblock}{Forts. Beispiel: Totale Wahrscheinlichkeit}
Gegeben seien
\begin{align*}
P(B|A_{1}) &=0.25,&P(B|A_{2}) &=0.10,& P(B|A_{3}) &=0.05\\
P(A_1) &=0.15,    &P(A_2)     &=0.65,& P(A_3) &=0.20 
\end{align*}
Wahrscheinlichkeit, dass die ausgew�hlte Person arbeitslos ist:
\uncover<2>{
\[ P(B)= 0.25\cdot 0.15+0.10\cdot 0.65+0.05\cdot 0.20= 0.1125 \]
}
\end{exampleblock}
\end{frame}

\begin{frame}
\frametitle{Wahrscheinlichkeit}
\framesubtitle{Satz von Bayes}
\begin{block}{Satz von Bayes}
Es gilt
\[ P(A|B)=\frac{P(B|A) P(A)}{P(B)} \]
\end{block}
\medskip
Regel zum Vertauschen von Bedingung und Bedingtem
\end{frame}

\begin{frame}
\frametitle{Wahrscheinlichkeit}
\framesubtitle{Satz von Bayes}
\begin{exampleblock}{Beispiel: Bayes}
Ereignisse:
\begin{align*}
A &:\text{ Kind studiert} &P(A)&=0.25\\
B &:\text{ "`Akademisches Elternhaus"'}&P(B)&=0.35
\end{align*}
Folgende bedingte Wahrscheinlichkeit wird berichtet:
\[ P(B|A) =0.60 \]
Wie hoch ist die interessantere Wahrscheinlichkeit  $P(A|B)$?
\end{exampleblock}
\end{frame}

\begin{frame}
\frametitle{Wahrscheinlichkeit}
\framesubtitle{Satz von Bayes}
\begin{exampleblock}{Forts. Beispiel: Bayes}
Die Wahrscheinlichkeit, dass das "`Kind studiert"' (A), wenn es aus einem 
"`akademischen Elternhaus"' (B) kommt, ist
\uncover<2>{
\begin{align*}
P(A|B)&= \frac{P(B|A)P(A)}{P(B)}\\
&= \frac{0.60\cdot 0.25}{0.35}\\
&= 0.4286
\end{align*}
}
\end{exampleblock}
\end{frame}

\begin{frame}
\frametitle{Wahrscheinlichkeit}
\framesubtitle{Unabh�ngigkeit}
\begin{block}{Unabh�ngigkeit}
Zwei Ereignisse $A$ und $B$ hei�en (stochastisch) unabh�ngig, wenn
\[ P(A\cap B) =P(A)\cdot P(B) \]
\end{block}
\begin{itemize}
\item (Fast) �quivalent: $A$ und $B$ sind unabh�ngig genau dann, wenn
$P(A|B)=P(A)$ bzw. $P(B|A)=P(B)$
\item Unabh�ngigkeit und Disjunktheit sind nicht dasselbe!
\end{itemize}
\end{frame}

\begin{frame}
\frametitle{Wahrscheinlichkeit}
\framesubtitle{Unabh�ngigkeit}
\begin{exampleblock}{Beispiel: Unabh�ngigkeit}
Ereignisse:
\begin{align*}
A &:\text{ Kind studiert} &P(A)&=0.25\\
B &:\text{ Akademisches Elternhaus}&P(B)&=0.35
\end{align*}
Gegeben ist auch $P(B|A) =0.60$. 
\medskip

Wegen $P(B|A)\neq P(B)$ sind $A$ und $B$ abh�ngig
\end{exampleblock}
\end{frame}

\end{document}