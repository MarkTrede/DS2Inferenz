\documentclass[12pt,show notes]{beamer} 
%\documentclass[12pt,handout]{beamer}

\usepackage{beamerthemesplit}
\usepackage{amsmath}
\usepackage{amsfonts}
\usepackage{graphicx}
\usepackage{color}
\usepackage{eurosym}
\usepackage[ngerman]{babel}
\usepackage[ansinew]{inputenc}

\usetheme{Rochester}
\usecolortheme{beaver}
\setbeamertemplate{navigation symbols}{} 
\setbeamertemplate{footline}[text line]{} 
\graphicspath{{../plots/}}

\begin{document}

\title{Data Science 2}
\subtitle{Zufallsvorg�nge}
\author{Prof.~Dr.~Mark Trede}
\institute[]{Institut f\"ur \"Okonometrie und Wirtschaftsstatistik}
\date{Oktober 2023}
\maketitle

\begin{frame}
\frametitle{Zufall}
\framesubtitle{Zufallsvorgang}
\begin{block}{Zufallsvorgang}
Ein Zufallsvorgang ist ein Vorgang, 
\begin{itemize}
\item bei dem im Voraus feststeht,
welche m�glichen Ergebnisse er haben kann,
\item das tats�chliche Ergebnis im Voraus jedoch nicht bekannt ist.
\end{itemize}
\end{block}
\medskip
Zufallsexperiment: kontrolliert wiederholbarer Zufallsvorgang
\end{frame}

\begin{frame}
\frametitle{Zufall}
\framesubtitle{Ergebnisraum}
\begin{block}{Ergebnisraum}
Die Menge aller Ergebnisse nennen wir Ergebnisraum\\
(oder Ergebnismenge) $\Omega$.
\end{block}
\begin{itemize}
\item Ein Element $\omega \in \Omega $ hei�t Ergebnis 
\item $|\Omega|$ ist die Anzahl der Elemente in $\Omega $
\item Beispiel: Werfen eines W�rfels, $\Omega =\{1,2,3,4,5,6\}$
\end{itemize}
\end{frame}

\begin{frame}
\frametitle{Zufall}
\framesubtitle{Ereignis}
\begin{block}{Ereignis}
Teilmengen von $\Omega$ hei�en Ereignisse.
\end{block}
\begin{itemize}
\item Notation: $A$, $B$, $C$, $\ldots $ oder $A_{1},A_{2},\ldots $
\item Ereignis $A$ tritt ein, falls $\omega \in A$
\item Ein Zufallsvorgang hat nur ein einziges Ergebnis,\\
aber es treten immer viele Ereignisse ein!
\item $\Omega$ hei�t sicheres Ereignis, $\varnothing$ hei�t unm�gliches Ereignis
\item Wenn $A\cap B=\varnothing$, hei�en $A$ und $B$ disjunkt
\end{itemize}
\end{frame}

\begin{frame}
\frametitle{Zufall}
\framesubtitle{Ereignis}
\begin{exampleblock}{Beispiel: W�rfelwurf}
Ergebnismenge $\Omega =\{1,2,3,4,5,6\}$
\begin{align*}
A &= \{2,4,6\},\text{ eine gerade Zahl w�rfeln}\\
B &= \{2\},\text{ eine 2 w�rfeln} \\
C &= \{1,2,3\},\text{ weniger als 4 w�rfeln} \\
D &= \{5,6\},\text{ mehr als 4 w�rfeln}
\end{align*}
Wird eine 2 gew�rfelt, treten $A$, $B$ und $C$ ein, aber $D$ nicht.
\end{exampleblock}
\end{frame}

\begin{frame}
\frametitle{Zufall}
\framesubtitle{Ereignis}
Verkn�pfungen von Ereignissen:\medskip
\begin{itemize}
\item $A\cap B$ tritt ein, wenn $A$ \textbf{und} $B$ eintreten
\item $A\cup B$ tritt ein, wenn $A$ \textbf{oder} $B$ eintritt
\item $\bar{A}$ tritt ein, wenn $A$ \textbf{nicht} eintritt
\item $A\backslash B$ tritt ein, wenn $A$ \textbf{ohne} $B$ eintritt, d.h.\ $A\cap\bar B$
\item $\bigcap_{i=1}^{n}A_{i}$ tritt ein, wenn \textbf{alle} $A_{i}$ eintreten
\item $\bigcup_{i=1}^{n}A_{i}$ tritt ein, wenn \textbf{mindestens ein} $A_{i}$ eintritt
\end{itemize}
\end{frame}

\begin{frame}
\frametitle{Zufall}
\framesubtitle{Ereignis}
\begin{exampleblock}{Beispiel: W�rfelwurf}
Ergebnismenge $\Omega =\{1,2,3,4,5,6\}$
\begin{align*}
A &= \{2,4,6\}\\
B &= \{1,2,3\}\\
A\cap B &=\{2\}\\
A\cup B&=\{1,2,3,4,6\}
\end{align*}
Wird eine 2 gew�rfelt, treten $A$, $B$, $A\cap B$ und $A\cup B$ ein.
\end{exampleblock}
\end{frame}

\begin{frame}
\frametitle{Zufall}
\framesubtitle{Ereignis}
\textbf{Rechenregeln f�r Mengen:}
\begin{itemize}
\item Distributivgesetz
\begin{eqnarray*}
A\cup (B\cap C) &=&(A\cup B)\cap(A\cup C)\\
A\cap (B\cup C) &=&(A\cap B)\cup(A\cap C)
\end{eqnarray*}
\item de Morgansche Regeln:
\begin{eqnarray*}
\overline{A\cup B} &=&\overline{A}\cap \overline{B} \\
\overline{A\cap B} &=&\overline{A}\cup \overline{B}
\end{eqnarray*}
\end{itemize}
\end{frame}

\end{document}