\documentclass{article}
\usepackage{amsmath}

\usepackage[a4paper]{geometry}
\usepackage[utf8]{inputenc}
\usepackage{amsmath}
\usepackage{amsfonts} 

\parindent0mm
\parskip1.5ex plus0.5ex minus0.5ex

\begin{document}
\section*{Lineare Transformation}

Sei $X$ eine Zufallsvariable mit Verteilungsfunktion $F_X(x)$.
Definiere $Y=aX+b$ mit $a,b\in\mathbb{R}$ und $a>0$. 
Dann gilt für die Verteilungsfunktion von $Y$
\begin{align*}
F_Y(y) &= P(Y\le y) \\
&= P(aX+b\le y)\\
&= P\left(X\le \frac{y-b}{a}\right)\\
&= F_X\left(\frac{y-b}{a}\right).
\end{align*}
Wenn $Y$ eine stetige Zufallsvariable ist, dann gilt für
die Dichte
\begin{align*}
f_Y(y) &= F'_Y(y) \\
&= F'_X\left(\frac{y-b}{a}\right)\cdot\frac{1}{a}\\
&= f_X\left(\frac{y-b}{a}\right)\cdot\frac{1}{a}.
\end{align*}

\section*{Varianz}
Aus
\[ V(X)=E((X-E(X))^2) \]
folgt
\begin{align*}
V(X) &= E[(X-E(X))^2] \\
&= E[X^2-2XE(X)+(E(X))^2] \\
&= E(X^2)-2E(X)E(X)+(E(X))^2 \\
&= E(X^2)-2(E(X))^2+(E(X))^2\\
&= E(X^2)-(E(X))^2.
\end{align*}

\section*{Varianz von linearen Transformationen}
Es gilt
\begin{align*}
V(aX+b) &= E[((aX+b)-E(aX+b))^2]\\
&= E[(aX+b-aE(X)-b)^2] \\
&= E[(aX-aE(X))^2]\\
&= E[a^2(X-E(X))^2]\\
&= a^2 E[(X-E(X))^2]\\
&= a^2 V(X).
\end{align*}
Die Standardabweichung ist also
\[ \sqrt{V(aX+b)} = |a|\sqrt{V(X)}. \]

\section*{Erwartungswert von $S^2$}
Wir betrachten den Varianzschätzer 
\[
S^{2}=\frac{1}{n}\sum_{i=1}^{n}\left( X_{i}-\bar{X}\right) ^{2}.
\]
Es gilt
\begin{eqnarray*}
E\left( S^{2}\right)  &=&E\left( \frac{1}{n}\sum_{i=1}^{n}\left( X_{i}-\bar{X%
}\right) ^{2}\right)  \\
&=&E\left( \frac{1}{n}\sum_{i=1}^{n}\left( X_{i}-\mu +\mu -\bar{X}\right)
^{2}\right)  \\
&=&E\left( \frac{1}{n}\sum_{i=1}^{n}\left[ \left( X_{i}-\mu \right)
^{2}+\left( \mu -\bar{X}\right) ^{2}+2\left( X_{i}-\mu \right) \left( \mu -%
\bar{X}\right) \right] \right)  \\
&=&E\left( \frac{1}{n}\sum_{i=1}^{n}\left( X_{i}-\mu \right) ^{2}+\frac{1}{n}%
\sum_{i=1}^{n}\left( \mu -\bar{X}\right) ^{2}-\frac{2}{n}\sum_{i=1}^{n}%
\left( X_{i}-\mu \right) \left( \bar{X}-\mu \right) \right)  \\
&=&\left( \frac{1}{n}\sum_{i=1}^{n}E\left( \left( X_{i}-\mu \right)
^{2}\right) \right) +E\left( \left( \bar{X}-\mu \right) ^{2}\right) -\frac{2%
}{n}\sum_{i=1}^{n}E\left( \left( X_{i}-\mu \right) \left( \bar{X}-\mu
\right) \right) .
\end{eqnarray*}%
Nebenrechnungen: Der erste Summand ist%
\begin{eqnarray*}
\frac{1}{n}\sum_{i=1}^{n}E\left( \left( X_{i}-\mu \right) ^{2}\right)  &=&%
\frac{1}{n}\sum_{i=1}^{n}V(X_{i}) \\
&=&\frac{1}{n}\sum_{i=1}^{n}\sigma ^{2} \\
&=&\sigma ^{2}.
\end{eqnarray*}%
Der zweite Summand ist 
\begin{eqnarray*}
E\left( \left( \bar{X}-\mu \right) ^{2}\right)  &=&V(\bar{X}) \\
&=&\sigma ^{2}/n.
\end{eqnarray*}%
Der dritte Summand ist%
\begin{eqnarray*}
\frac{2}{n}\sum_{i=1}^{n}E\left( \left( X_{i}-\mu \right) \left( \bar{X}-\mu
\right) \right)  &=&2E\left( \left( X_{1}-\mu \right) \left( \bar{X}-\mu
\right) \right)  \\
&=&2Cov(X_{1},\bar{X}) \\
&=&2Cov\left( X_{1},\frac{1}{n}\sum_{i=1}^{n}X_{i}\right)  \\
&=&\frac{2}{n}\sum_{i=1}^{n}Cov(X_{1},X_{i}) \\
&=&\frac{2}{n}\sigma ^{2},
\end{eqnarray*}%
da $Cov(X_{1},X_{1})=V(X_{1})=\sigma ^{2}$ und $Cov(X_{1},X_{i})=0$ f\"{u}r $%
i>1.$ F\"{u}hrt man die drei Terme wieder zusammen, ergibt sich%
\begin{eqnarray*}
E\left( S^{2}\right)  &=&\sigma ^{2}+\frac{\sigma ^{2}}{n}-\frac{2\sigma ^{2}%
}{n} \\
&=&\sigma ^{2}-\frac{\sigma ^{2}}{n} \\
&=&\frac{n-1}{n}\sigma ^{2}.
\end{eqnarray*}%

\section*{Konfidenzintervalle}
Ausgangspunkt der Konfidenzintervalle f\"{u}r Erwartungswerte 
ist die Verteilung des standardisierten Stichprobenmittels.

Wenn die Stichprobe aus einer Normalverteilung stammt, gilt f\"{u}r jeden
Stichprobenumfang $n$%
\begin{eqnarray*}
\sqrt{n}\frac{\bar{X}-\mu }{\sigma } &\sim &N(0,1) \\
\sqrt{n}\frac{\bar{X}-\mu }{S^{\ast }} &\sim &t_{n-1}.
\end{eqnarray*}

Wenn die Stichprobe nicht aus einer Normalverteilung stammt, m\"{u}ssen wir
den zentralen Grenzwertsatz anwenden, d.h. der Stichprobenumfang $n$ muss groß sein. 
Dann gilt%
\begin{eqnarray*}
&&\sqrt{n}\frac{\bar{X}-\mu }{\sigma }\overset{appr}{\sim }N(0,1) \\
&&\sqrt{n}\frac{\bar{X}-\mu }{S^{\ast }}\overset{appr}{\sim }N(0,1).
\end{eqnarray*}

F\"{u}r Konfidenzintervalle werden Quantile der $N(0,1)$- oder der $t_{n-1}$%
-Verteilung bestimmt, so dass%
\begin{eqnarray*}
P\left( -c\leq \sqrt{n}\frac{\bar{X}-\mu }{\sigma }\leq c\right) 
&=&1-\alpha  \\
P\left( -c\leq \sqrt{n}\frac{\bar{X}-\mu }{S^{\ast }}\leq c\right) 
&=&1-\alpha ;
\end{eqnarray*}%
anschlie\ss end l\"{o}st man die Ungleichungen in der Wahrscheinlichkeit
nach $\mu $ auf.

Aulösen nach $\mu$ an einem Beispiel:
\begin{align*}
P\left( -c\leq \sqrt{n}\frac{\bar{X}-\mu }{\sigma }\leq c\right) &= 1-\alpha \\
P\left( -c\frac{\sigma}{\sqrt{n}}\leq (\bar{X}-\mu)\leq c\frac{\sigma}{\sqrt{n}}\right) &= 1-\alpha \\
P\left( -\bar X-c\frac{\sigma}{\sqrt{n}}\leq -\mu \leq -\bar X+c\frac{\sigma}{\sqrt{n}}\right) &= 1-\alpha \\
P\left( \bar X+c\frac{\sigma}{\sqrt{n}}\geq \mu \geq \bar X-c\frac{\sigma}{\sqrt{n}}\right) &= 1-\alpha \\
P\left( \bar X-c\frac{\sigma}{\sqrt{n}}\leq \mu \leq \bar X+c\frac{\sigma}{\sqrt{n}}\right) &= 1-\alpha 
\end{align*}
Also ist das Konfidenzintervall
\[
\bar X\pm c\frac{\sigma}{\sqrt{n}}.
\]
Analog geht man für die anderen Fälle vor.

\section*{Hypothesentests}
Ausgangspunkt der Erwartungswerttests ist die Verteilung 
des standardisierten Stichprobenmittels.
Bei Hypothesentests ist unter G\"{u}ltigkeit der Nullhypothese $\mu =\mu_0$. 
Die Teststatistik lautet
\begin{eqnarray*}
T &=&\sqrt{n}\frac{\bar{X}-\mu_0}{\sigma } \\
T &=&\sqrt{n}\frac{\bar{X}-\mu_0}{S^{\ast }}.
\end{eqnarray*}
Der kritische Wert ist dann ein geeignetes Quantil der $N(0,1)$- oder der 
$t_{n-1}$-Verteilung.

\section*{Powerfunktion}
Die Powerfunktion gibt in Abhängigkeit vom Parameter $\mu$ an, wie groß die 
Wahrscheinlichkeit ist, dass die Teststatistik $T$ im kritischen Bereich liegt.
Wir betrachten beispielhaft den rechtsseitigen Gaußtest mit den Hypothesen
\begin{align*}
H_0 &: \mu\le\mu_0\\
H_1 &: \mu > \mu_0.
\end{align*}
Der kritische Bereich des rechtsseitigen Gaußtests ist rechts vom
kritischen Wert $u_{1-\alpha}$. Die Wahrscheinlichkeit, dass die
Teststatistik im kritischen Bereich liegt, ist also
\begin{align*}
P\left(\sqrt{n}\frac{\bar X-\mu_0}{\sigma}>u_{1-\alpha}\right) &=
P\left(\sqrt{n}\frac{\bar X-\mu+\mu-\mu_0}{\sigma}>u_{1-\alpha}\right)\\
&=P\left(\sqrt{n}\frac{\bar X-\mu}{\sigma}+\sqrt{n}\frac{\mu-\mu_0}{\sigma}>u_{1-\alpha}\right)\\
&=P\left(\sqrt{n}\frac{\bar X-\mu}{\sigma}>u_{1-\alpha}-\sqrt{n}\frac{\mu-\mu_0}{\sigma}\right)\\
&=1-P\left(\sqrt{n}\frac{\bar X-\mu}{\sigma}\le u_{1-\alpha}-\sqrt{n}\frac{\mu-\mu_0}{\sigma}\right)\\
&=1-\Phi\left(u_{1-\alpha}-\sqrt{n}\frac{\mu-\mu_0}{\sigma}\right).
\end{align*}

Für das Beispiel (Folie 379) ergibt sich als Power an $\mu=201$
\begin{align*}
P\left(\sqrt{n}\frac{\bar X-\mu_0}{\sigma}>u_{1-\alpha}\right) &=
1-\Phi\left(u_{1-\alpha}-\sqrt{n}\frac{\mu-\mu_0}{\sigma}\right)\\
&=1-\Phi\left(1.6449-\sqrt{8}\frac{201-198}{2}\right)\\
&=1-\Phi(-2.60)\\
&=\Phi(2.60)\\
&=0.9953.
\end{align*}


\end{document}
