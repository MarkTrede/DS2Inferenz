\section{Grenzwerts�tze}

\begin{frame}
\frametitle{Grenzwerts�tze}
\framesubtitle{Kap.~3.2}
\textbf{Grenzwerts�tze}\medskip
\begin{itemize}
\item Gegeben sei eine Folge von unabh�ngigen, identisch
verteilten Zufallsvariablen (i.i.d.)
\[ X_{1},X_{2},X_{3},\ldots , \]
\item Die Folgenelemente $X_{1},X_{2},\ldots $ hei�en auch 
unabh�ngige Wiederholungen von $X$
\end{itemize}
\end{frame}

\begin{frame}
\frametitle{Grenzwerts�tze}
\framesubtitle{Kap.~3.2}
\begin{itemize}
\item F�r gegebenes $n$ ist das arithmetische Mittel
\[ \bar{X}_{n}=\frac{1}{n}\sum_{i=1}^{n}X_{i} \]
\item Achtung: $\bar{X}_{n}$ ist eine \textbf{Zufallsvariable}!
\end{itemize}
\end{frame}

\begin{frame}
\frametitle{Grenzwerts�tze}
\framesubtitle{Kap.~3.2}
\begin{itemize}
\item Was passiert mit der Verteilung von $\bar{X}_n$, 
wenn $n\rightarrow \infty $ geht?
\item Sei $E(X)=\mu $ und $Var(X)=\sigma ^{2}$
\item Dann gilt
\begin{eqnarray*}
E(\bar{X}_{n}) &=&\mu  \\
Var(\bar{X}_{n}) &=&\frac{\sigma ^{2}}{n}
\end{eqnarray*}
\end{itemize}
\end{frame}

\begin{frame}
\frametitle{Grenzwerts�tze}
\framesubtitle{Kap.~3.2}
$E(\bar{X}_n)=$ \\
\vspace*{3cm}
$Var(\bar{X}_{n}) =$\\
\vspace*{2cm}
\end{frame}

\begin{frame}
\frametitle{Grenzwerts�tze}
\framesubtitle{Kap.~3.2}
\begin{block}{Schwaches Gesetz der gro�en Zahl}
F�r jedes (noch so kleine) $\varepsilon >0$ gilt
\[ \lim_{n\rightarrow \infty}P(|\bar{X}_{n}-\mu|\geq \varepsilon) =0. \]
\end{block}
Alternative Schreibweise
\[ \textnormal{plim}_{n\rightarrow \infty }\bar{X}_{n}=\mu \]
\end{frame}

\begin{frame}
\frametitle{Grenzwerts�tze}
\framesubtitle{Kap.~3.2}
\begin{itemize}
\item Anschaulich: Die Verteilung von $\bar{X}_{n}$ zieht sich immer mehr
auf $\mu $ zusammen
\item Spezialfall: $X$ sei Bernoulli-verteilt mit Parameter $P(X=1)=\pi $
\item Dann ist $\bar{X}_{n}$ die relative H�ufigkeit der Erfolge
\item Wegen $E(X)=\pi $ gilt
\[ \lim_{n\rightarrow \infty }P\left( \left\vert \bar{X}_{n}-\pi \right\vert
\geq \varepsilon \right) =0 \]
\end{itemize}
\end{frame}

\begin{frame}
\frametitle{Grenzwerts�tze}
\framesubtitle{Kap.~3.2}
\begin{exampleblock}{Beispiel: Ein W�rfel wird geworfen}
Sei 
\[ X=\left\{ 
\begin{array}{ll}
0 & \quad \text{wenn Augenzahl nicht 5 ist} \\ 
1 & \quad \text{wenn Augenzahl 5 ist}
\end{array}\right. \]
Es gilt
\[ P(X=1)=\frac{1}{6} \]
\end{exampleblock}\end{frame}

\begin{frame}
\frametitle{Grenzwerts�tze}
\framesubtitle{Kap.~3.2}
\begin{exampleblock}{Forts. Beispiel: Ein W�rfel wird geworfen}
\begin{itemize}
\item Der W�rfel wird nun sehr oft geworfen ($n$ Mal)
\item $X_{1},X_{2},\ldots $ geben jeweils an, ob eine 5 geworfen wurde
\item $\bar{X}_{n}$ ist der Anteil der F�nfen
\item F�r gro�es $n$ geht $\bar{X}_{n}$ gegen $1/6$
\item Simulation in R: [\texttt{wlln.R}]
\end{itemize}
\end{exampleblock}
\end{frame}

\begin{frame}
\frametitle{Grenzwerts�tze}
\framesubtitle{Kap.~3.2}
\begin{exampleblock}{Beispiel: Produktgewicht}
\begin{itemize}
\item Die Zufallsvariable $X\sim N(201,4)$
sei das tats�chliche Gewicht einer 200g-Tafel Schokolade
\item $X_i$ ist das Gewicht der $i$-ten Tafel, $i=1,2,\ldots $
\item $\bar{X}_{n}$ ist das Durchschnittsgewicht dieser Tafeln
\item F�r gro�es $n$ geht $\bar{X}_n$ gegen 201\hfill [\texttt{wlln.R}]
\end{itemize}
\end{exampleblock}
\end{frame}


\begin{frame}
	\frametitle{Grenzwerts�tze}
	\framesubtitle{Shiny - Schwaches Gesetz der gro�en Zahl}
	\begin{figure}
		\includegraphics[height=\textheight]{shiny-logo}
	\end{figure}
\end{frame}


\begin{frame}
\frametitle{Grenzwerts�tze}
\framesubtitle{Kap.~3.2}
\begin{itemize}
\item Zentraler Grenzwertsatz (Begr�ndung f�r die 
extreme Wichtigkeit der Normalverteilung)
\item Definiere das standardisierte arithmetische Mittel
\[ U_{n}=\frac{\bar{X}_{n}-\mu}{\sqrt{\sigma^2/n}}
=\sqrt{n}\frac{\bar{X}_n-\mu}{\sigma}\]
\end{itemize}
\end{frame}

\begin{frame}
\frametitle{Grenzwerts�tze}
\framesubtitle{Kap.~3.2}
\begin{block}{Zentraler Grenzwertsatz}
F�r alle $u\in \mathbb{R}$ gilt
\[ \lim_{n\rightarrow \infty }P(U_n\leq u)=\Phi(u), \]
wobei $\Phi$ die Verteilungsfunktion der $N(0,1)$ ist
\end{block}
F�r gro�es $n$ gilt approximativ
\[ U_n\overset{appr}{\sim }N(0,1) \]
\end{frame}

\begin{frame}
\frametitle{Grenzwerts�tze}
\framesubtitle{Kap.~3.2}
\begin{itemize}
\item Folglich gilt auch
\[ \sum_{i=1}^{n}X_{i}\overset{appr}{\sim }N(n\mu ,n\sigma ^{2}) \]
und
\[ \bar{X}_{n}\overset{appr}{\sim }N\left( \mu ,\frac{\sigma ^{2}}{n}\right) \]
\end{itemize}
\end{frame}

\begin{frame}
\frametitle{Grenzwerts�tze}
\framesubtitle{Kap.~3.2}
\begin{itemize}
\item Die Summe und der Durchschnitt von $n$ 
\textbf{beliebig verteilten} Zufallsvariablen ist 
approximativ normalverteilt, wenn $n$ gro� genug ist!
\item Es gibt einige einschr�nkende Bedingungen, aber 
in den meisten Situationen gilt der zentrale Grenzwertsatz
\item Simulationen in R
\end{itemize}
\end{frame}

\begin{frame}
\frametitle{Grenzwerts�tze}
\framesubtitle{Kap.~3.2}
\begin{itemize}
\item Spezialfall: $X$ Bernoulli-verteilt mit $\pi$
\begin{eqnarray*}
E(X) &=&\pi  \\
V(X) &=&\pi \left( 1-\pi \right) 
\end{eqnarray*}
und daher
\[ \sum_{i=1}^{n}X_i\overset{appr}{\sim }N(n\pi,n\pi (1-\pi))\]
\end{itemize}
\end{frame}

\begin{frame}
\frametitle{Grenzwerts�tze}
\framesubtitle{Kap.~3.2}
\begin{itemize}\item Approximation der Binomialverteilung durch die 
Normalverteilung (De Moivre, 1733)
\item Wegen 
\[ \sum_{i=1}^{n}X_i\overset{appr}{\sim }N(n\pi ,n\pi (1-\pi)) \]
gilt
\[ P\left( \sum_{i=1}^{n}X_{i}\leq b\right) 
\approx \Phi \left( \frac{b-n\pi }{\sqrt{n\pi \left( 1-\pi \right) }}\right) \]
\end{itemize}
\end{frame}

\begin{frame}
\frametitle{Grenzwerts�tze}
\framesubtitle{Kap.~3.2}
\begin{itemize}
\item Verbesserung der Approximationen durch eine Stetigkeitskorrektur
\[ P\left( \sum_{i=1}^{n}X_{i}\leq b\right) 
\approx \Phi \left( \frac{b+0.5-n\pi }{\sqrt{n\pi (1-\pi)}}\right) \]
bzw.
\[ P\left( \sum_{i=1}^{n}X_{i}<b\right) 
\approx \Phi \left( \frac{b-0.5-n\pi }{\sqrt{n\pi (1-\pi)}}\right) \]
\end{itemize}
\end{frame}

\begin{frame}
\frametitle{Grenzwerts�tze}
\framesubtitle{Kap.~3.2}
\begin{exampleblock}{Beispiel: Marketing}
Eine Marketing-Abteilung verschickt an $n=500$ zuf�llig ausgew�hlte 
Kunden Frageb�gen.\medskip

$X_{i}$ sind Bernoulli-verteilt mit Parameter $\pi =0.2$
\begin{equation*}
X_{i}=\left\{ 
\begin{array}{ll}
1 & \quad \text{wenn Kunde }i\text{ antwortet} \\ 
0 & \quad \text{wenn Kunde }i\text{ nicht antwortet}
\end{array}
\right. 
\end{equation*}
Sei $Y=\sum_{i=1}^{500}X_{i}$ die Zahl der Antworten.
\end{exampleblock}
\end{frame}

\begin{frame}
\frametitle{Grenzwerts�tze}
\framesubtitle{Kap.~3.2}
\begin{exampleblock}{Forts. Beispiel: Marketing}
Die Wahrscheinlichkeit, dass zwischen 95 und 105 Kunden antworten, ist
\vspace*{4cm}
\end{exampleblock}
\end{frame}


\begin{frame}
	\frametitle{Grenzwerts�tze}
	\framesubtitle{Shiny - Zentraler Grenzwertsatz}
	\begin{figure}
		\includegraphics[height=\textheight]{shiny-logo}
	\end{figure}
\end{frame}


