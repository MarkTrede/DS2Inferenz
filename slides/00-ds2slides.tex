\documentclass[12pt,show notes]{beamer} 
%\documentclass[12pt,handout]{beamer}

\usepackage{beamerthemesplit}
\usepackage{amsmath}
\usepackage{amsfonts}
\usepackage{graphicx}
\usepackage{color}
\usepackage{eurosym}
\usepackage[ngerman]{babel}
\usepackage[ansinew]{inputenc}

\usetheme{Rochester}
\usecolortheme{beaver}
\setbeamertemplate{navigation symbols}{} 
\setbeamertemplate{footline}[text line]{} 
\graphicspath{{../plots/}}

\begin{document}

\title{Data Science 2}
\author{Prof.~Dr.~Mark Trede}
\institute[]{Institut f\"ur \"Okonometrie und Wirtschaftsstatistik}
\date{Wintersemester 2023/2024}
\maketitle

\begin{frame}
\frametitle{Einleitung}
\textbf{Data Science 2: Statistische Inferenz}\\\medskip
\begin{enumerate}
\item Grundlagen der Wahrscheinlichkeitstheorie\\
{\footnotesize Zufall, Wahrscheinlichkeit, Zufallsvariablen, Standardverteilungen,
Zufallsvektoren, Kovarianz}
\item Grenzwerts�tze und Computersimulationen
\item Statistische Inferenz\\
{\footnotesize Stichproben, Punkt- und Intervallsch�tzung, Hypothesentests}
\end{enumerate}
\end{frame}

\begin{frame}
\frametitle{Einleitung}
\textbf{Wozu braucht man Wahrscheinlichkeitstheorie?}\\\medskip
\begin{itemize}
\item Wahrscheinlichkeitstheorie ist ein Teil der\\  �konomischen Theorie
\item Verhalten unter Unsicherheit (deskriptiv, normativ)
\item Wahrscheinlichkeitstheorie ist das formale Fundament 
f�r die statistische Inferenz
\end{itemize}
\end{frame}

\begin{frame}
\frametitle{Einleitung}
\textbf{Wozu braucht man statistische Inferenz?}\\\medskip
\begin{itemize}
\item Sch�tzung �konomischer Gr��en aus nicht perfekter Datenbasis (Stichproben, verrauschte Daten, Zeitreihen)
\item Quantifizierung der Pr�zision von Sch�tzungen
\item Testen �konomischer Theorien (Hypothesen)
\end{itemize}
\end{frame}

\begin{frame}
\frametitle{Organisation des Moduls}
\textbf{Organisation:}\medskip
\begin{itemize}
\item Vorlesung
\item Training: �bungen, Hausaufgaben,``Klickaufgaben''
\item Hilfen und Materialien
\item Pr�fung\bigskip\pause
\item[$\longrightarrow$]
Alle Informationen und Materialien finden Sie auf der Learnweb-Seite
dieses Moduls
\end{itemize}
\end{frame}

\begin{frame}
\frametitle{Organisation des Moduls}
\textbf{Organisation: Vorlesung}\medskip
\begin{itemize}
\item Do., 14-16 c.t., Aula am Aasee
\item Kein Streaming
\item Keine Aufzeichnung
\item Terminplan
\end{itemize}
\end{frame}

\begin{frame}
\frametitle{Organisation des Moduls}
\textbf{Organisation: Training}\medskip
\begin{itemize}
\item Acht �bungsbl�tter: Bearbeitung in Saal�bungen (``Tutorien''), L�sungsskizzen
\item F�nf Hausaufgaben: Gruppenarbeit, L�sungsvideos
\item ``Klickaufgaben'' im Learnweb: (fast) jede Woche
\item Bonuspunkte f�r die Klausur, wenn Sie
alle Hausaufgaben und fast alle Klickaufgaben bestehen
\end{itemize}
\end{frame}

\begin{frame}
\frametitle{Organisation des Moduls}
\textbf{Organisation: Hilfen und Materialien}\medskip
\begin{itemize}
\item eLehrbuch (\texttt{bookdown.org/marktrede/ds2inferenz})
\item Datens�tze
\item Fragestunde 
\end{itemize}
\end{frame}

\begin{frame}
\frametitle{Organisation des Moduls}
\textbf{Organisation: Pr�fung}\medskip
\begin{itemize}
\item Klausur (120 Minuten, Papier-und-Stift)
\item Der Termin wird vom Pr�fungsamt festgelegt
\item Bonuspunkte, wenn alle Hausaufgaben und fast alle Klickaufgaben 
bestanden wurden
\item Details im Learnweb
\end{itemize}
\end{frame}

\begin{frame}
\frametitle{Organisation des Moduls}
\textbf{Voraussetzungen:}\medskip
\begin{itemize}
\item Data Science 1 ist \textbf{keine} Voraussetzung
\item Integral- und Differentialrechnung
\item Schulmathematik zur Wahrscheinlichkeitsrechnung
\item R und RStudio
\end{itemize}
\end{frame}

\begin{frame}
\frametitle{Organisation des Moduls}
\textbf{Nutzung von R}\medskip
\begin{itemize}
\item Erst etwas sp�ter im Semester
\item Verwendung von R-Funktionen (z.B. Quantilfunktionen)
\item Computer-Simulationen f�r ein tieferes Verst�ndnis der Methoden
\item Programmierung (z.B. Schleifen)
\end{itemize}
\end{frame}

\begin{frame}
\frametitle{Organisation des Moduls}
\textbf{To Do's}\medskip
\begin{enumerate}
\item Tragen Sie sich im Learnweb f�r einen �bungstermin ein\\
($\longrightarrow$ Zuweisung zu den Saal�bungen)
\item Tragen Sie sich im Learnweb f�r die Gruppenzuteilung ein\\
($\longrightarrow$ Zuordnung zu einer Gruppe f�r die Hausaufgaben)
\item Verlinken Sie das eLehrbuch 
{\footnotesize (\texttt{bookdown.org/marktrede/ds2inferenz})}
\item Installieren Sie R und RStudio
{\footnotesize (\texttt{bookdown.org/marktrede/DS1Deskription/rstudio.html})}
\end{enumerate}
\end{frame}

\end{document}